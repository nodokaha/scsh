%% $Id: install-lib.tex,v 1.5 2005/08/20 15:46:29 michel-schinz Exp $

\documentclass[a4paper,12pt]{article}

\usepackage[latin1]{inputenc}
\usepackage{a4wide, palatino, url, hyperref}

\newcommand{\file}{\begingroup \urlstyle{tt}\Url}

\newcommand{\envvar}[1]{\texttt{#1}}
\newcommand{\cloption}[1]{\texttt{#1}}
\newcommand{\package}[1]{\texttt{#1}}
\newcommand{\layout}[1]{\texttt{#1}}
\newcommand{\location}[1]{\texttt{#1}}
\newcommand{\ident}[1]{\texttt{#1}}
\newcommand{\phase}[1]{\texttt{#1}}

\newcommand{\define}[3]{%
  \noindent%
  (\texttt{#1} \textit{#2})\hfill\textit{(#3)}\\[0.5em]%
}
\newcommand{\definep}[2]{\define{#1}{#2}{procedure}}
\newcommand{\defines}[2]{\define{#1}{#2}{syntax}}
\newcommand{\param}[1]{\emph{#1}}

\newenvironment{rationale}%
{\begin{quotation}\noindent\textbf{Rationale}}%
{\end{quotation}}
\newenvironment{example}%
{\begin{quotation}\noindent\textbf{Example}}%
{\end{quotation}}

\begin{document}
\title{A proposal for scsh packages}
\author{Michel Schinz}
\maketitle

\section{Introduction}
\label{sec:introduction}

The aim of the following proposal is to define a standard for the
packaging, distribution, installation, use and removal of libraries
for scsh. Such packaged libraries are called \emph{scsh packages} or
simply \emph{packages} below.

This proposal attempts to cover both libraries containing only Scheme
code and libraries containing additional C code. It does not try to
cover applications written in scsh, which are currently considered to
be outside of its scope.

\subsection{Package identification and naming}

Packages are identified by a globally-unique name. This name should
start with an ASCII letter (a-z or A-Z) and should consist only of
ASCII letters, digits or hyphens `\verb|-|'. Package names are
case-sensitive, but there should not be two packages with names which
differ only by their capitalisation.

\begin{rationale}
  This restriction on package names ensures that they can be used to
  name directories on current operating systems.
\end{rationale}

Several versions of a given package can exist. A version is identified
by a sequence of non-negative integers. Versions are ordered
lexicographically.

A version has a printed representation which is obtained by separating
(the printed representation of) its components by dots. For example,
the printed representation of a version composed of the integer 1
followed by the integer 2 is the string \texttt{1.2}. Below, versions
are usually represented using their printed representation for
simplicity, but it is important to keep in mind that versions are
sequences of integers, not strings.

A specific version of a package is therefore identified by a name and
a version. The \emph{full name} of a version of a package is obtained
by concatenating:
\begin{itemize}
\item the name of the package,
\item a hyphen `\texttt{-}',
\item the printed representation of the version.
\end{itemize}

In what follows, the term \emph{package} is often used to designate a
specific version of a package, but this should be clear from the
context.

\section{Distributing packages}

Packages are usually distributed as a stand-alone archive containing
the code and documentation to install, as well as a file describing
the installation procedure. However, it is also possible to distribute
the scsh-specific installation procedure separately from the main code
and documentation. This is typically useful to turn general Scheme
libraries into scsh packages.

\subsection{Stand-alone packages}
\label{sec:stand-alone-packages}

Stand-alone packages are distributed in a single \texttt{tar} archive,
which can optionally be compressed by \texttt{gzip} or \texttt{bzip2}.
The name of the archive is composed by appending:
\begin{itemize}
\item the full name of the package,
\item the string \texttt{.tar} indicating that it's a \texttt{tar}
  archive,
\item either the string \texttt{.gz} if the archive is compressed
  using \texttt{gzip}, or the string \texttt{.bz2} if the archive is
  compressed using \texttt{bzip2}, or nothing if the archive is not
  compressed.
\end{itemize}

The archive of a stand-alone package is organised so that it contains
one top-level directory whose name is the full name of the package.
This directory is called the \emph{package unpacking directory}. All
the files belonging to the package are stored below it.

The unpacking directory contains at least the following files:
\begin{description}
\item[\file{pkg-def.scm}] a Scheme file containing the installation
  procedure for the package (see �~\ref{sec:authoring-packages}),
\item[\file{README}] a textual file containing a short description of
  the package,
\item[\file{COPYING}] a textual file containing the license of the
  package.
\end{description}

\subsection{Split packages}
\label{sec:split-packages}

Split packages are distributed as two separate archives. The first
one, called the \emph{upstream archive}, contains the code and
documentation of the package. The second one, called the
\emph{downstream archive}, contains the scsh-specific installation
procedure for the package.

Ideally, the upstream archive should follow the rules laid out in the
previous section for stand-alone archives. Most of the time, however,
the author of the upstream package has no reason to be aware of the
current policy, or simply to adhere to it. The name and layout of the
downstream archive should therefore mirror the one of the upstream
archive, as follows:
\begin{enumerate}
\item the name of the downstream archive is composed by appending:
  \begin{itemize}
  \item the string \texttt{pkg\_} (the three letters `p', `k' and `g'
    followed by an underscore character),
  \item the name of the upstream archive, including its version but
    excluding extensions indicating the archive type and/or
    compression method,
  \item an underscore character,
  \item a single number indicating the version of the downstream
    archive,
  \item the string \texttt{.tar} indicating that it's a \texttt{tar}
    archive,
  \item either the string \texttt{.gz} if the archive is compressed
    using \texttt{gzip}, or the string \texttt{.bz2} if the archive is
    compressed using \texttt{bzip2}, or nothing if the archive is not
    compressed.
  \end{itemize}
\item the downstream archive should contain at least the file
  \file{pkg-def.scm} placed in a directory with the same name as the
  ``main'' directory of the upstream archive.
\end{enumerate}
These conventions ensure that split packages behave almost like
stand-alone packages after the two archives have been unpacked at the
same location.

% TODO give an example of a split package

\section{Downloading and installing packages}

A package can be installed on a target machine by downloading its
archive, expanding it and finally running the installation script
located in the unpacking directory.

\subsection{Layouts}

The installation script installs files according to some given
\emph{layout}. A layout maps abstract \emph{locations} to concrete
directories on the target machine. For example, a layout could map the
abstract location \location{doc}, where documentation is stored, to
the directory \file{/usr/local/share/doc/my_package}.

Currently, the following abstract locations are defined:
\begin{description}
\item[\location{base}] The ``base'' location of a package, where the
  package loading script \file{load.scm} resides.

\item[\location{active}] Location containing a symbolic link, with the
  same name as the package (excluding the version), pointing to the
  base location of the package.  This link is used to designate the
  \emph{active} version of a package\,---\,the one to load when a
  package is requested by giving only its name, without an explicit
  version.

\item[\location{scheme}] Location containing all Scheme code. If the
  package comes with some examples showing its usage, they are put in
  a sub-directory called \file{examples} of this location.

\item[\location{lib}] Location containing platform-dependent files,
  like shared libraries.  This location contains one sub-directory per
  platform for which packages have been installed, and nothing else.

\item[\location{doc}] Location containing all the package
  documentation. This location contains one or more sub-directories,
  one per format in which the documentation is available. The contents
  of these sub-directories is standardised as follows, to make it easy
  for users to find the document they need:
  \begin{description}
  \item[\file{html}] Directory containing the HTML documentation of
    the package, if any; this directory should at least contain one
    file called \file{index.html} serving as an entry point to the
    documentation.
  \item[\file{pdf}] Directory containing the PDF documentation of the
    package, if any; this directory should contain at least one file
    called \file{<package_name>.pdf} where \file{<package_name>} is
    the name of the package.
  \item[\file{ps}] Directory containing the PostScript documentation
    of the package, if any; this directory should contain at least one
    file called \file{<package_name>.ps} where \file{<package_name>}
    is the name of the package.
  \item[\file{text}] Directory containing the raw textual
    documentation of the package, if any.
  \end{description}

\item[\location{misc-shared}] Location containing miscellaneous data
  which does not belong to any directory above, and which is
  platform-independent.
\end{description}

The directories to which a layout maps these abstract locations are
not absolute directories, but rather relative ones. They are relative
to a \emph{prefix}, specified at installation time using the
\cloption{--prefix} option, as explained in section
\ref{sec:inst-proc}.

\begin{example}
  Let's imagine that a user is installing version 1.2 of a package
  called \package{foo}. This package contains a file called
  \file{COPYING} which has to be installed in sub-directory
  \file{license} of the \location{doc} location. If the user chooses
  to use the default layout, which maps \location{doc} to directory
  \file{<scsh-version>/<package_full_name>/doc} (see
  �~\ref{sec:scsh-layout}), and specifies
  \file{/usr/local/share/scsh/modules} as a prefix, then the
  \file{COPYING} file will end up in:
\[
\begin{small}
\underbrace{\mbox{\texttt{/usr/local/share/scsh/modules/}}}_{1}%
\underbrace{\mbox{\texttt{0.6/foo-1.2/doc/}}}_{2}%
\underbrace{\mbox{\texttt{license/COPYING}}}_{3}
\end{small}
\]
(provided the user is running scsh v0.6.x) Part 1 is the prefix, part
2 is the layout's mapping for the \location{doc} location, and part 3
is the file name relative to the location.
\end{example}

\subsubsection{Predefined layouts}
\label{sec:predefined-layouts}

Every installation script comes with a set of predefined layouts which
serve different aims. They are described below

The directories to which these layouts map locations often have a name
which includes the current version of scsh and/or the full name of the
package. In what follows, the notation \file{<version>} represents the
printed representation of the first two components of scsh's version
(e.g. \file{0.6} for scsh v0.6.x). The notation \file{<pkg_fname>}
represents the \emph{full} name of the package being installed.

\paragraph{The \layout{scsh} layout}
\label{sec:scsh-layout}

The \layout{scsh} layout is the default layout. It maps all locations
to sub-directories of a single directory, called the package
installation directory, which contains nothing but the files of the
package being installed. Its name is simply the full name of the
package in question, and it resides in the \file{prefix} directory.

The \layout{scsh} layout maps locations as given in the following
table:
\begin{center}
  \begin{tabular}{|l|l|}
    \hline
    \textbf{Location}      & \textbf{Directory (relative to prefix)} \\
    \hline
    \location{base}        & \file{<version>/<pkg_fname>}            \\
    \location{active}      & \file{<version>}                        \\
    \location{scheme}      & \file{<version>/<pkg_fname>/scheme}     \\
    \location{lib}         & \file{<version>/<pkg_fname>/lib}        \\
    \location{doc}         & \file{<version>/<pkg_fname>/doc}        \\
    \location{misc-shared} & \file{<version>/<pkg_fname>}            \\
    \hline
  \end{tabular}
\end{center}

This layout is well suited for installations performed without the
assistance of an additional package manager, because it makes many
common operations easy. For example, finding to which package a file
belongs is trivial, as is the removal of an installed package.

\paragraph{The \layout{fhs} layout}
\label{sec:fhs-layout}

The \layout{fhs} layout maps locations according to the File Hierarchy
Standard (FHS, see \href{http://www.pathname.com/fhs/}%
{http://www.pathname.com/fhs/}), as follows:
\begin{center}
  \begin{tabular}{|l|l|}
    \hline
    \textbf{Location}    & \textbf{Directory (relative to prefix)}                \\
    \hline
    \layout{base}        & \file{share/scsh-<version>/modules/<pkg_fname>}        \\
    \layout{active}      & \file{share/scsh-<version>/modules}                    \\
    \layout{scheme}      & \file{share/scsh-<version>/modules/<pkg_fname>/scheme} \\
    \layout{lib}         & \file{lib/scsh-<version>/modules/<pkg_fname>}          \\
    \layout{doc}         & \file{share/doc/scsh-<version>/<pkg_fname>}            \\
    \layout{misc-shared} & \file{share/scsh-<version>/modules/<pkg_fname>}        \\
    \hline
  \end{tabular}
\end{center}

The main advantage of this layout is that it adheres to the FHS
standard, and is therefore compatible with several packaging policies,
like \href{http://www.debian.org/}{Debian}'s,
\href{http://fink.sourceforge.net/}{Fink}'s and others. Its main
drawback is that files belonging to a given package are scattered, and
therefore hard to find when removing or upgrading a package. Its use
should therefore be considered only if third-party tools are available
to track files belonging to a package.

%% TODO \subsection{File permissions}

\subsection{Installation procedure}
\label{sec:inst-proc}

Packages are installed using the \file{scsh-install-pkg} script,
which is part of the installation library. This script must be given
the name of the prefix using the \cloption{--prefix} option. It also
accepts the following options:
\begin{center}
  \begin{tabular}{lp{.6\textwidth}}
    \cloption{--layout} name      & Specifies the layout to use (see �~\ref{sec:predefined-layouts}).                         \\
    \cloption{--verbose}          & Print messages about what is being done.                                                  \\
    \cloption{--dry-run}          & Print what actions would be performed to install the package, but do not perform them.    \\
    \cloption{--inactive}         & Do not activate package after installing it.                                              \\
    \cloption{--non-shared-only}  & Only install platform-dependent files, if any.                                            \\
    \cloption{--force}            & Overwrite existing files during installation.                                             \\
    \cloption{--no-user-defaults} & Don't read user defaults in \file{.scsh-pkg-defaults.scm} (see �~\ref{sec:user-preferences}). \\
  \end{tabular}
\end{center}
A few, more advanced options exist and are documented in section
\ref{sec:packaging-packages}.

\subsubsection{User preferences}
\label{sec:user-preferences}

Users can store default values for the options passed to the
installation script by storing them in a file called
\file{.scsh-pkg-defaults.scm} residing in their home directory. This
file must contain exactly one Scheme expression whose value is an
association list. The keys of this list, which must be symbols,
identify options and the values specify the default value for these
options. The contents of this file is implicitly quasi-quoted.

The values stored in this file override the default values of the
options, but they are in turn overridden by the values specified on the
command line of the installation script. Furthermore, it is possible
to ask for this file to be completely ignored by passing the
\cloption{--no-user-defaults} option to the installation script.

\begin{example}
  A \file{.scsh-pkg-defaults.scm} file containing the following:
\begin{verbatim}
;; Default values for scsh packages installation
((layout  . "fhs")
 (prefix  . "/usr/local/share/scsh/modules")
 (verbose . #t))
\end{verbatim}
  specifies default values for the \cloption{--layout},
  \cloption{--prefix} and \cloption{--verbose} options.
\end{example}

\section{Using packages}

To use a package, its \emph{loading script} must be loaded in
Scheme~48's exec package. The loading script for a package is a file
written in the Scheme 48 exec language, whose name is \file{load.scm}
and which resides in the \location{base} location.

To load this file, one typically uses scsh's \cloption{-lel} option
along with a properly defined \envvar{SCSH\_LIB\_DIRS} environment
variable.

Scsh has a list of directories, called the library directories, in
which it looks for files to load when the options \cloption{-ll} or
\cloption{-lel} are used. This list can be given a default value
during scsh's configuration, and this value can be overridden by
setting the environment variable \envvar{SCSH\_LIB\_DIRS} before running
scsh.

In order for scsh to find the package loading scripts, one must make
sure that scsh's library search path contains the names of all
\location{active} locations which containing packages.

The names of these directories should not end with a slash `\verb|/|',
as this forces scsh to search them recursively. This could
\emph{drastically} slow down scsh when looking for packages.

\begin{example}
  Let's imagine a machine on which the system administrator installs
  scsh packages according to the \layout{fhs} layout in prefix
  directory \file{/usr/local}. The \location{active} location for
  these packages corresponds to the directory
  \file{/usr/local/share/scsh-0.6/modules}, according to section
  \ref{sec:fhs-layout}.

  Let's also imagine that there is a user called \texttt{john} on this
  machine, who installs additional scsh packages for himself in his
  home directory, using \file{/home/john/scsh} as a prefix.
  To ease their management, he uses the \layout{scsh} layout. The
  \location{active} location for these packages corresponds to the
  directory \file{/home/john/scsh/0.6}, according to section
  \ref{sec:scsh-layout}.

  In order to be able to use scsh packages installed both by the
  administrator and by himself, user \texttt{john} needs to put both
  active directories in his \envvar{SCSH\_LIB\_DIRS} environment
  variable.  The value of this variable will therefore be:
  \begin{small}
\begin{verbatim}
"/usr/local/share/scsh-0.6/modules" "/home/john/scsh/0.6"
\end{verbatim}
  \end{small}

  Now, in order to use packages \package{foo} and \package{bar} in one
  of his script, user \texttt{john} just needs to load their loading
  script using the \cloption{-lel} option when invoking scsh, as
  follows:
\begin{verbatim}
  -lel foo/load.scm -lel bar/load.scm
\end{verbatim}
\end{example}

\section{Authoring packages}
\label{sec:authoring-packages}

Once the Scheme and/or C code for a package has been written, the last
step in turning it into a standard package as defined by this proposal
is to write the installation script.

This script could be written fully by the package author, but in order
to simplify this task a small scsh installation framework is provided.
This framework must be present on the host system before a scsh
package can be installed.

As explained above, when the \file{scsh-install-pkg} script is
invoked, it launches scsh on the main function of the installation
library, which does the following:
\begin{enumerate}
\item parse the command line arguments (e.g the \cloption{--prefix}
  option),
\item load the package definition file, a (Scheme) file called
  \file{pkg-def.scm}, which is supplied by the package author and
  which contains one or several package definition statements, and
\item install the packages which were defined in the previous step.
\end{enumerate}
Most package definition files should contain a single package
definition, but the ability to define several packages in one file can
sometimes be useful.

The main job of the package author is therefore to write the package
definition file, \file{pkg-def.scm}. This file is mostly composed of a
package definition statement, which specifies the name, version and
installation code for the package. The package definition statement is
expressed using the \ident{define-package} form, documented in the
next section.

\subsection{Installation library}
\label{sec:install-library}

\subsubsection{Package definition}

\defines{define-package}{name version extension body ...}%
Define a package to be installed. \param{Name} (a string) is the
package name, \param{version} (a list of integers) is its version,
\param{extensions} is a list of extensions (see below), and
\param{body} is the list of statements to be evaluated in order to
install the package.

The installation statements typically use functions of the
installation library in order to install files in their target
location. The available functions are presented below.

\param{Extensions} consists in a list of lists, each one starting with
a symbol identifying the extension, possibly followed by
extension-specific parameters. It is used to specify various
parameters, which are usually optional. Currently, the following
extensions are defined:
\begin{description}
\item[install-lib-version] specifies the version of the installation
  library that this package definition requires. The version is
  specified as a list composed of \emph{exactly two} integers, giving
  the major and minor version number of the library. Before installing
  a package, this version requirement is checked and installation
  aborts if the installation library does not satisfy it.%
  \footnote{Version $(i_1\,i_2)$ of the installation library satisfies a
    requirement $(r_1\,r_2)$ if and only if both major numbers are equal,
    and the minor number of the installation library is greater or
    equal to the minor requirement. In other words, iff $i_1 = r_1$
    and $i_2 \ge r_2$.}%
  It is strongly recommended that package authors provide this
  information, as it makes it possible to provide helpful error
  messages to users.

\item[options] enables the script to define additional command-line
  options. It accepts nine parameters in total, with the last three
  being optional. The description of these parameters follows, in the
  order in which they should appear:
  \begin{description}
  \item[\param{name}] (a symbol) is the name of the option, without
    the initial double hyphen (\verb|--|),
  \item[\param{help-text}] (a string) describes the option for the
    user,
  \item[\param{arg-help-text}] (a string) describes the option's
    argument (if any) for the user,
  \item[\param{required-arg?}] (a boolean) says whether this option
    requires an argument or not,
  \item[\param{optional-arg?}] (a boolean) says whether this option's
    argument can be omitted or not,
  \item[\param{default}] (anything) is the default value for the
    option,
  \item[\param{parser}] (a function from string to anything) parses
    the option, i.e. turns its string representation into its internal
    value,
  \item[\param{unparser}] (a function from anything to string) turns
    the internal representation of the option into a string,
  \item[\param{transformer}] is a function taking the current value of
    the option, the value given by the user and returning its new
    value.
  \end{description}
  By default, \param{parser} and \param{unparser} are the identity
  function, and \param{transformer} is a function which takes two
  arguments and returns the second (i.e. the current value of the
  option is simply replaced by the one given).
\end{description}

\subsubsection{Content installation}

\definep{install-file}{file location [target-dir]}%
Install the given \param{file} in the sub-directory \param{target-dir}
(which must be a relative directory) of the given \param{location}.
\param{Target-dir} is \file{.} by default.

If the directory in which the file is about to be installed does not
exist, it is created along with all its parents, as needed. If
\param{file} is a string, then the installed file will have the same
name as the original one. If \param{file} is a pair, then its first element
specifies the name of the source file, and its second element the name
it will have once installed. The second element must be a simple file
name, without any directory part.

\vspace{1em}
\definep{install-files}{file-list location [target-dir]}%
Like \ident{install-file} but for several files, which are specified
as a list. Each element in the list can be either a simple string or a
pair, as explained above.

\vspace{1em}
\definep{install-directory}{directory location [target-dir]}%
Install the given \param{directory} and all its contents, including
sub-directories, in sub-directory \param{target-dir} of
\param{location}. This is similar to what \param{install-file} does,
but for complete hierarchies.

Notice that \param{directory} will be installed as a sub-directory of
\param{target-dir}.

\vspace{1em}
\definep{install-directories}{dir-list location [target-dir]}%
Install several directories in one go.

\vspace{1em}
\definep{install-directory-contents}{directory location [target-dir]}%
Install the contents of the given \param{directory} in sub-directory
\param{target} of \param{location}.

\vspace{1em}
\definep{install-string}{string location [target-dir]}%
Install the contents of \param{string} in sub-directory
\param{target-dir} of \param{location}.

\subsubsection{Queries}

\definep{get-directory}{location install?}%
Get the absolute name of the directory to which the current layout
maps the abstract \param{location}. If \param{install?} is true, the
directory is the one valid during installation; If it is false, the
directory is the one valid after installation, that is when the
package is later used.

The distinction between installation-time and usage-time directories
is necessary to support staged installation, as performed by package
managers like Debian's APT.

\vspace{1em}
\definep{get-option-value}{option}%
Return the value of the given command-line \param{option} (a symbol).
This can be used to get the value of predefined options (like
\cloption{--dry-run}) or package-specific options.

\vspace{1em} \definep{phase-active?}{phase}%
Return true iff the given \param{phase} is active, that is if the
steps associated with it should be performed. \param{Phase} should be
one of the following symbols: \texttt{build}, \texttt{build-clean} or
\texttt{install}, designating the corresponding phase (see section
\ref{sec:packaging-packages}).

\subsubsection{Load script generation}

\definep{with-output-to-load-script*}{thunk}%
Evaluate \param{thunk} with the current output opened on the loading
script of the current package. If this script was already existing,
its previous contents is deleted.

\vspace{1em}
\defines{with-output-to-load-script}{body ...}%
Syntactic sugar for \ident{with-output-to-load-script*}.

\vspace{1em}
\definep{write-to-load-script}{s-expression}%
Pretty-print the \param{s-expression} to the loading script of the
current package. If this script was already existing, its previous
contents is deleted.

\begin{example}
  A typical package definition file for a simple package called
  \package{pkg} whose version is 1.2 could look like this:
\begin{verbatim}
(define-package "pkg" (1 2) ()
  (install-file "load.scm" 'base)
  (install-directory-contents "scheme" 'scheme)
  (install-file ("LICENSE" . "COPYING") 'doc)
  (install-directory-contents "doc" 'doc))
\end{verbatim}
  
  With such a definition, invoking the installation script with
  \file{/usr/local/} as prefix and \layout{fhs} as layout has the
  following effects:
\begin{enumerate}
\item The base directory \file{/usr/local/share/scsh/modules/pkg-1.2}
  is created and file \file{load.scm} is copied to it.
\item All the contents of the directory called \file{scheme} is copied
  to directory \file{/usr/local/share/scsh/modules/pkg-1.2/scheme}
  which is created before, if needed.
\item File \file{LICENSE} is copied to directory
  \file{/usr/local/share/doc/pkg-1.2/} with name \file{COPYING}.
\item All the contents of the directory called \file{doc} is copied to
  directory \file{/usr/local/share/doc/pkg-1.2/}
\item The package is activated by creating a symbolic link with name
  \file{/usr/local/share/scsh/modules/pkg} pointing to
  \file{./pkg-1.2}
\end{enumerate}
\end{example}

\subsubsection{Miscellaneous}
\label{sec:misc}

A few functions which are not specifically related to installation are
provided, as they can sometimes be useful when writing installation
scripts. They are documented below.

\vspace{1em}
\definep{parent-directory}{dir}%
Returns the parent directory of \param{dir}. Notice that a trailing
slash is always ignored by this function, so that the parent directory
of both \file{/tmp/dir} and \file{/tmp/dir/} is \file{/tmp/}.

\vspace{1em}
\definep{create-directory\&parents}{dir}%
Similar to scsh's \ident{create-directory}, but also create all parent
directories which do not exist yet.

\vspace{1em}
\definep{relative-file-name}{fname [dir]}%
Return the name of file \param{fname} relative to \param{dir}, which
defaults to the current directory.

\vspace{1em}
\definep{paths->file-name}{path ...}%
Similar to scsh's \ident{path-list->file-name} except that all
arguments are taken to form the final path. This function has no
equivalent for \ident{path-list->file-name}'s optional argument.

\subsection{Packages containing C code (for shared libraries)}

Packages containing C code are more challenging to write, since all
the problems related to C's portability and incompatibilities between
the APIs of the various platforms have to be accounted for.
Fortunately, the GNU Autoconf system simplifies the management of
these problems, and authors of scsh packages containing C code are
strongly encouraged to use it.

%% Integrating Autoconf into the installation procedure should not be a
%% major problem thanks to scsh's ability to run separate programs.

\section{Packaging packages}
\label{sec:packaging-packages}

Most important Unix systems today have one (or several) package
management systems which ease the installation of packages on a
system. In order to avoid confusion between these packages and the
scsh packages discussed above, they will be called \emph{system
  packages} in what follows.

It makes perfect sense to provide system packages for scsh packages.
System packages should as much as possible try to use the standard
installation script described above to install scsh packages. In order
to make this as easy as possible, the installation script recognizes
the following additional options:

\begin{center}
  \begin{tabular}{lp{.6\textwidth}}
    \cloption{--dest-dir} dir  & Specify a prefix for all installed files\\
    \cloption{--phases} phases & Only perform some phases of installation.\\
  \end{tabular}
\end{center}

The \cloption{--dest-dir} option plays the same role as the
\envvar{DESTDIR} variable which is typically given to \texttt{make
  install}. Its value, which is the empty string by default, is
prepended to all installation paths. This is used to perform staged
installations, during which the package is first installed into a
temporary directory and then moved to its final location by some
external tool.

The \cloption{--phases} option is used to perform only some steps of
the whole installation process. Currently, three phases are defined:
\phase{build}, \phase{build-clean} and \phase{install}.

The \phase{build} phase is the one during which the operations
required to build the package are performed. For scsh packages
containing only Scheme code, this phase usually does nothing. For scsh
packages containing some C code, this phase is the one during which
the C code is compiled.

The \phase{build-clean} phase is the one during which all the files
created during the \phase{build} phase are removed.

The \phase{install} phase is the one during which the files are copied
to their final location.

Using the \cloption{--phases} option, each phase can happen during a
separate run of the installation script. This makes it possible, for
example, to perform the build phase as one user, and the install phase
as another one (e.g. root). In order for this to work, the
installation script must be invoked as follows:
\begin{enumerate}
\item first with the \cloption{--phases build} option and maybe other
  options,
\item then with the \emph{exact same options} as in the first step,
  except that \phase{install} has to be given to \cloption{--phases}
  instead of \phase{build}.
\end{enumerate}
The behaviour of the installation script is not defined if these
conditions are violated.

%% \section{Glossary}
%% TODO define the following terms
%% Version
%% Target machine
%% Package
%% (Package) unpacking directory
%% Layout
%% (Abstract) location
%% Package loading script

\section{Acknowledgments}
\label{sec:acknowledgments}

Discussions with
Andreas Bernauer,
Anthony Carrico,
David Frese,
Friedrich Delgado Friedrichs,
Martin Gasbichler,
Eric Knauel,
Daniel Kobras,
Lionel Elie Mamane, and
Jorgen Sch�fer
greatly helped the design of this proposal.
Mark Sapa started everything by asking for a Fink package for sunet
and sunterlib.

\end{document}