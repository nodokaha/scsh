\chapter{Time and Daytime}\label{cha:ntp}

Many Unix hosts provide a RFC~867 Daytime service which sends the
current date and time as a human-readable character string. The
daytime service is typically served on port 13 as both TCP and UDP.

The RFC~868 Time protocol provides a site-independent, machine
readable date and time.  The Time service is typically served
on port 37 as TCP and UDP. The idea is that you can confirm your
system's idea of the time by polling several independent sites on the
network.

\section{Daytime}

The \ex{rfc867} structure contains an interface to Daytime protocol.

\defun{rfc867-daytime/tcp}{host}{string}
\defunx{rfc867-daytime/udp}{host [timeout-or-\sharpf]}{string-or-\sharpf}
\begin{desc}
  These procedures asks \var{host} about the current daytime and
  return the host's answer (e.g., ``Thursday, April 4,
  2'').
  
  \ex{Rfc867-daytime/tcp} uses the TCP variant of the protocol.
  \ex{Rfc867-daytime/udp} uses UDP and sends a single request to the
  server.  It allows the specification of an optional timeout; if not
  specified or \sharpf{}, \ex{Rfc867-daytime/udp} will wait
  indefinitely for an answer.  If the answer from the server doesn't
  arrive within the specified time, \ex{rfc867-daytime/udp} returns
  \sharpf.
\end{desc}

\section{Time}

The \ex{rfc868} structure contains an interface to the Time protocol.

\defun{rfc868-time/tcp}{host}{string}
\defunx{rfc868-time/udp}{host [timeout-or-\sharpf]}{string-or-\sharpf}
\begin{desc}
  These procedures asks \var{host} about the current time and return
  the host's answer.  This is the number of second since 1970, just as
  with scsh's \texttt{time} procedure.

  \ex{rfc868-time/tcp} uses the TCP variant of the protocol.
  \ex{rfc868-time/udp} uses UDP and sends a single request to the
  server.  It allows the specification of an optional timeout; if not
  specified or \sharpf{}, \ex{rfc868-time/udp} will wait
  indefinitely for an answer.  If the answer from the server doesn't
  arrive within the specified time, \ex{rfc868-time/udp} returns
  \sharpf.
\end{desc}


%%% Local Variables: 
%%% mode: latex
%%% TeX-master: "man"
%%% End: 
