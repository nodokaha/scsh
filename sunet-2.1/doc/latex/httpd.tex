\chapter{HTTP server}\label{cha:httpd}
%
The SUnet HTTP Server is a complete industrial-strength implementation
of the HTTP 1.0 protocol.  It is highly configurable and allows the writing
of dynamic web pages that run inside the server without going through
complicated and slow protocols like CGI or Fast/CGI.
       
\section{Starting and configuring the server}

All procedures described in this section are exported by the
\texttt{httpd} structure.

The Web server is started by calling the \ex{httpd} procedure, which takes
one argument, an options value:

\defun{httpd}{options}{\noreturn}
\begin{desc}
  This procedure starts the server. The \var{options} argument
  specifies various configuration parameters, explained below.
  
  The server's basic loop is to wait on the port for a connection from
  an HTTP client. When it receives a connection, it reads in and
  parses the request into a special request data structure.  Then the
  server forks a thread which binds the current I/O ports to the
  connection socket, and then hands off to the top-level
  request handler (which must be specified in the options). The
  request handler is responsible for actually serving
  the request---it can be any arbitrary computation.  Its output goes
  directly back to the HTTP client that sent the request.
   
  Before calling the request handler to service the request, the HTTP
  server installs an error handler that fields any uncaught error,
  sends an error reply to the client, and aborts the request
  transaction.  Hence any error caused by a request handler will be
  handled in a reasonable and robust fashion.
\end{desc}
%
The \var{options} argument can be constructed through a number of procedures
with names of the form \texttt{with-\ldots}.  Each of these procedures
either creates a fresh options value or adds a configuration parameter
to an old options argument.  The configuration parameter value is
always the first argument, the (old) options value the optional second
one.  Here they are:

\defun{with-port}{port [options]}{options}
\begin{desc}
  This specifies the port on which the server listens.  Defaults to 80.
\end{desc}

\defun{with-root-directory}{root-directory [options]}{options} 
\begin{desc}
  This specifies the current directory of the server.  Note that this
  is \emph{not} the document root directory.  Defaults to \texttt{/}.
\end{desc}

\defun{with-fqdn}{fqdn [options]}{options}
\begin{desc}
  This specifies the fully-qualified domain name the server uses in
  automatically generated replies, or \ex{\#f} if the server should
  query DNS for the fully-qualified domain name..  Defaults to \ex{\#f}.
\end{desc}

\defun{with-reported-port}{reported-port [options]}{options}
\begin{desc}
  This specifies the port number the server uses in automatically
  generated replies or \ex{\#f} if the reported port is the same as
  the port the server is listening on.  (This is useful if you're
  running the server through an accelerating proxy.)  Defaults to
  \ex{\#f}.
\end{desc}

\defun{with-server-admin}{mail-address [options]}{options}
\begin{desc}
  This specifies the email address of the server administrator the
  server uses in automatically generated replies.  Defaults to \ex{\#f}.
\end{desc}

\defun{with-request-handler}{request-handler [options]}{options}
\begin{desc}
  This specifies the request handler of the server to which the server
  delegates the actual work.  More on that subject below in
  Section~\ref{httpd:request-handlers}.  This parameter must be specified.
\end{desc}

\defun{with-simultaneous-requests}{requests [options]}{options}
\begin{desc}
  This specifies a limit on the number of simultaneous requests the
  server servers.  If that limit is exceeded during operation, the
  server will hold off on new requests until the number of
  simultaneous requests has sunk below the limit again.  If this
  parameter is \ex{\#f}, no limit is imposed.  Defaults to \ex{\#f}.
\end{desc}

\defun{with-log-file}{log-file [options]}{options}
\begin{desc}
  This specifies the name of a log file for the server where it writes
  Common Log Format logging information.  It can also be a port in
  which case the information is logged to that port, or \ex{\#f} for
  no logging.  Defaults to \ex{\#f}.
  
  To allow rotation of log files, the server re-opens the log file
  whenever it receives a \texttt{USR1} signal.
\end{desc}

\defun{with-syslog?}{syslog? [options]}{options}
\begin{desc}
  This specifies whether the server will log information about
  incoming to the Unix syslog facility.  Defaults to \ex{\#t}.
\end{desc}

\defun{with-resolve-ip?}{resolve-ip? [options]}{options}
\begin{desc}
  This specifies whether the server writes the domain names rather
  than numerical IPs to the output log it produces.  Defaults to
  \ex{\#t}.
\end{desc}

To avoid paranthitis, the \ex{make-httpd-options} procedure eases the
construction of the options argument:

\defun{make-httpd-options}{transformer value \ldots}{options}
\begin{desc}
  This constructs an options value from an argument list of parameter
  transformers and parameter values.  The arguments come in pairs,
  each an option transformer from the list above, and a value for that
  parameter.  \ex{Make-httpd-options} returns the resulting options value.
\end{desc}

For example,
\begin{alltt}
(httpd (make-httpd-options
         with-request-handler (rooted-file-handler "/usr/local/etc/httpd")
         with-root-directory "/usr/local/etc/httpd"))
\end{alltt}
  %
  starts the server on port 80 with
  \ex{/usr/local/etc/httpd} as its root directory and
  lets it serve any file out from this directory.
  % #### note about rooted-file-handler


\section{Requests}
\label{httpd:requests}

Request handlers operate on \textit{requests} which contain the
information needed to generate a page.  The relevant procedures to
dissect requests are defined in the \texttt{httpd-requests} structure:

\defun{request?}{value}{boolean}
\defunx{request-method}{request}{string}
\defunx{request-uri}{request}{string}
\defunx{request-url}{request}{url}
\defunx{request-version}{request}{pair}
\defunx{request-headers}{request}{list}
\defunx{request-socket}{request}{socket}
\begin{desc}
  The procedure inspect request values.  \ex{Request?} is a predicate
  for requests.  \ex{Request-method} extracts the method of the HTTP
  request; it's a string such as \verb|"GET"|, \verb|"PUT"|.
  \ex{Request-uri} returns the escaped URI string as read from request
  line.  \ex{Request-url} returns an HTTP URL value (see the
  description of the \ex{url} structure in \ref{cha:url}).
  \ex{Request-version} returns \verb|(major . minor)| integer pair
  representing the version specified in the HTTP request.
  \ex{Request-headers} returns an association lists of header field
  names and their values, each represented by a list of strings, one
  for each line.  \ex{Request-socket} returns the socket connected
  to the client.\footnote{Request handlers should not perform I/O on the
    request record's socket.  Request handlers are frequently called
    recursively, and doing I/O directly to the socket might bypass a
    filtering or other processing step interposed on the current I/O ports
    by some superior request handler.}
\end{desc}

\section{Responses}
\label{sec:http-responses}

A path handler must return a \textit{response} value representing the
content to be sent to the client.  The machinery presented here for
constructing responses lives in the \ex{httpd-responses} structure.

\defun{make-response}{status-code maybe-message seconds mime extras
  body}{response}
\begin{desc}
  This procedure constructs a response value.  \var{Status-code} is an
  HTTP status code (more on that below).  \var{Maybe-message} is a a
  message elaborating on the circumstances of the status code; it can
  also be \sharpf{} meaning that the server should send a default
  message associated with the status code.  \var{Seconds} natural
  number indicating the time the content was created, typically the
  value of \verb|(time)|.  \var{Mime} is a string indicating the MIME
  type of the response (such as \verb|"text/html"| or
  \verb|"application/octet-stream"|).  \var{Extras} is an association
  list with extra headers to be added to the response; its elements
  are pairs, each of which consists of a symbol representing the field
  name and a string representing the field value.  \var{Body}
  represents the body of the response; more on that below.
\end{desc}

\defun{make-redirect-response}{location}{response}
\begin{desc}
  This is a helper procedure for constructing HTTP redirections.  The
  server will serve the new file indicated by \var{location}.
  \var{Location} must be URI-encoded and begin with a slash.
\end{desc}

\defun{make-error-response}{status-code request [message] extras \ldots}{response}
\begin{desc}
  This is a helper procedure for constructing error responses.
  \var{code} is status code of the response (see below). \var{Request}
  is the request that led to the error.  \var{Message} is an optional
  string containing an error message written in HTML, and \var{extras}
  are further optional arguments containing further message lines to
  be added to the web page that's generated.

  \ex{Make-error-response} constructs a response value which generates
  a web page containg a short explanatory message for the error at hand.
\end{desc}

\begin{table}[htb]
  \centering
  \begin{tabular}{|l|l|l|}
    \hline
   ok &                 200 & OK\\\hline
   created &            201 & Created\\\hline
   accepted &           202 & Accepted\\\hline
   prov-info &          203 & Provisional Information\\\hline
   no-content &         204 & No Content\\\hline

   mult-choice &        300 & Multiple Choices\\\hline
   moved-perm &         301 & Moved Permanently\\\hline
   moved-temp &         302 & Moved Temporarily\\\hline
   method &             303 & Method (obsolete)\\\hline
   not-mod &            304 & Not Modified\\\hline

   bad-request &        400 & Bad Request\\\hline
   unauthorized &       401 & Unauthorized\\\hline
   payment-req &        402 & Payment Required\\\hline
   forbidden &          403 & Forbidden\\\hline
   not-found &          404 & Not Found\\\hline
   method-not-allowed & 405 & Method Not Allowed\\\hline
   none-acceptable &    406 & None Acceptable\\\hline
   proxy-auth-required & 407 & Proxy Authentication Required\\\hline
   timeout &            408 & Request Timeout\\\hline
   conflict &           409 & Conflict\\\hline
   gone &               410 & Gone\\\hline
   internal-error &     500 & Internal Server Error\\\hline
   not-implemented &    501 & Not Implemented\\\hline
   bad-gateway &        502 & Bad Gateway\\\hline
   service-unavailable & 503 & Service Unavailable\\\hline
   gateway-timeout &    504 & Gateway Timeout\\\hline
  \end{tabular}
  \caption{HTTP status codes}
  \label{tab:status-code-names}
\end{table}

\dfn{status-code}{\synvar{name}}{status-code}{syntax}
\defunx{name->status-code}{symbol}{status-code}
\defunx{status-code-number}{status-code}{integer}
\defunx{status-code-message}{status-code}{string}
\begin{desc}
  The \ex{status-code} syntax returns a status code where
  \synvar{name} is the name from Table~\ref{tab:status-code-names}.
  \ex{Name->status-code} also returns a status code for a name
  represented as a symbol.  For a given status code,
  \ex{status-code-number} extracts its number, and
  \ex{status-code-message} extracts its associated default message.
\end{desc}

\section{Response Bodies}
\label{httpd:response-bodies}

A \textit{response body} represents the body of an HTTP response.
There are several types of response bodies, depending on the
requirements on content generation.

\defun{make-writer-body}{proc}{body}
\begin{desc}
  This constructs a response body from a \textit{writer}---a procedure
  that prints the page contents to a port.  The \var{proc} argument
  must be a procedure accepting an output port (to which \var{proc}
  prints the body) and the options value passed to the \ex{httpd}
  invocation.
\end{desc}

\defun{make-reader-writer-body}{proc}{body}
\begin{desc}
  This constructs a response body from a \textit{reader/writer}---a
  procedure that prints the page contents to a port, possibly after
  reading input from the socket of the HTTP connection.  The
  \var{proc} argument must be a procedure accepting three arguments:
  an input port (associated with the HTTP connection socket), an
  output port (to which \var{proc} prints the body), and the options
  value passed to the \ex{httpd} invocation.
\end{desc}

\section{Request Handlers}
\label{httpd:request-handlers}

A request handler generates the actual content for a request; request
handlers form a simple algebra and may be combined and composed in
various ways.


A request handler is a procedure of two arguments like this:
\defun{request-handler}{path req}{response}
\begin{desc}
  \var{Req} is a request. The \semvar{path} argument is the URL's
  path, parsed and split at slashes into a string list. For example,
  if the Web client dereferences URL
  %
\begin{verbatim}
http://clark.lcs.mit.edu:8001/h/shivers/code/web.tar.gz
\end{verbatim}
  then the server would pass the following path to the top-level
  handler:
  %
\begin{verbatim}
("h" "shivers" "code" "web.tar.gz")
\end{verbatim}
  %
  The \var{path} argument's pre-parsed representation as a string
  list makes it easy for the request handler to implement recursive
  operations dispatch on URL paths.

  The request handler must return an HTTP response.
\end{desc}

\subsection{Basic Request Handlers}
 
The web server comes with a useful toolbox of basic request handlers
that can be used and built upon.  The following procedures are
exported by the \ex{httpd\=basic\=handlers} structure:

\defvar{null-request-handler}{request-handler}
\begin{desc}
  This request handler always generated a \ex{not-found} error
  response, no patter what the request is.
\end{desc}

\defun{make-predicate-handler}{predicate handler
  default-handler}{request-handler}
\begin{desc}
  The request handler returned by this procedure first calls
  \var{predicate} on its path and request; it then acts like
  \var{handler} if the predicate returned a true vale, and like
  \var{default-handler} if the predicate returned \sharpf.
\end{desc}

\defun{make-host-name-handler}{hostname handler default-handler}{request-handler}
\begin{desc}
  The request handler returned by this procedure compares the host
  name specified in the request with \var{hostname}: if they match, it
  acts like \var{handler}, otherwise, it acts like
  \var{default-handler}.
\end{desc}

\defun{make-path-predicate-handler}{predicate handler
  default-handler}{request-handler}
\begin{desc}
  The request handler returned by this procedure first calls
  \var{predicate} on its path; it then acts like \var{handler} if the
  predicate returned a true vale, and like \var{default-handler} if
  the predicate returned \sharpf.
\end{desc}

\defun{make-path-prefix-handler}{path-prefix handler default-handler}{request-handler}
\begin{desc}
  This constructs a request handler that calls \var{handler} on its
  argument if \var{path-prefix} (a string) is the first element of the
  requested path; it calls \var{handler} on the rest of the path and
  the original request.  Otherwise, the handler acts like
  \var{default-handler}.
\end{desc}

\defun{alist-path-dispatcher}{handler-alist default-handler}{request-handler}
\begin{desc}
  This procedure takes as arguments an alist mapping strings to path
  handlers, and a default request handler, and returns a handler that
  dispatches on its path argument.  When the new request handler is
  applied to a path 
\begin{verbatim}
("foo" "bar" "baz")
\end{verbatim}
  it uses the
  first element of the path---\ex{foo}---to index into the
  alist.  If it finds an associated request handler in the alist, it
  hands the request off to that handler, passing it the tail of the
  path, in this case
\begin{verbatim}
("bar" "baz")
\end{verbatim}
  %
  On the other hand, if the path is
  empty, or the alist search does not yield a hit, we hand off to the
  default path handler, passing it the entire original path,
\begin{verbatim}
("foo" "bar" "baz")
\end{verbatim}
  %
  This procedure is how you say: ``If the first element of the URL's
  path is `foo', do X; if it's `bar', do Y; otherwise, do Z.''
  The slash-delimited URI path structure implies an associated tree of
  names. The request-handler system and the alist dispatcher allow you to
  procedurally define the server's response to any arbitrary subtree
  of the path space.
          
  Example: A typical top-level request handler is
\begin{alltt}          
(define ph
  (alist-path-dispatcher
      `(("h"       . ,(home-dir-handler "public\_html"))
        ("cgi-bin" . ,(cgi-handler "/usr/local/etc/httpd/cgi-bin"))
        ("seval"   . ,seval-handler))
      (rooted-file-handler "/usr/local/etc/httpd/htdocs")))
\end{alltt}
    
    This means:
\begin{itemize}          
\item If the path looks like \ex{("h"\ob{} "shivers"\ob{}
    "code"\ob{} "web.\ob{}tar.\ob{}gz")}, pass the path
  \ex{("shivers"\ob{} "code"\ob{} "web.\ob{}tar.\ob{}gz")} to a
  home-directory request handler.
\item If the path looks like \ex{("cgi-\ob{}bin"\ob{} "calendar")},
    pass ("calendar") off to the CGI request handler.
  \item If the path looks like \ex{("seval"\ob{} \ldots)}, the tail
    of the path is passed off to the code-uploading \ex{seval} path
    handler.
  \item Otherwise, the whole path is passed to a rooted file handler,
    who will convert it into a filename, rooted at
    \ex{/usr/\ob{}lo\ob{}cal/\ob{}etc/\ob{}httpd/\ob{}htdocs},
    and serve that file.
\end{itemize}
\end{desc}

\subsection{Static Content Request Handlers}

The request handlers described in this section are for serving static
content off directory trees in the file system.  They live in the
\ex{httpd-file-directory-handlers} structure.

The request handlers in this section eventually call an internal
procedure named \ex{file\=serve} for serving files which implements a
simple directory-generation service using the following rules:
\begin{itemize}
\item If the filename has the form of a directory (i.e., it ends with
  a slash), then \ex{file\=serve} actually looks for a file named
  \ex{index.html} in that directory.
\item If the filename names a directory, but is not in directory form
  (i.e., it doesn't end in a slash, as in
  ``\ex{/usr\ob{}in\ob{}clu\ob{}de}'' or ``\ex{/usr\ob{}raj}''),
  then \ex{file\=serve} sends back a ``301 moved permanently''
  message, redirecting the client to a slash-terminated version of the
  original URL. For example, the URL
  \ex{http://\ob{}clark.\ob{}lcs.\ob{}mit.\ob{}edu/\ob{}~shi\ob{}vers}
  would be redirected to
  \ex{http://\ob{}clark.\ob{}lcs.\ob{}mit.\ob{}edu/\ob{}~shi\ob{}vers/}
\item If the filename names a regular file, it is served to the
  client.
\end{itemize}
%
The \ex{httpd-file-directory-handlers} all take an options value as an
argument, similar to the options for \ex{httpd} itself.

The \var{options} argument can be constructed through a number of procedures
with names of the form \texttt{with-\ldots}.  Each of these procedures
either creates a fresh options value or adds a configuration parameter
to an old options argument.  The configuration parameter value is
always the first argument, the (old) options value the optional second
one.  Here they are:

\defun{with-file-name->content-type}{proc [options]}{options}
\begin{desc}
  This specifies a procedure for determining the MIME content type
  (``\ex{text/html},'' ``\ex{application/octet-stream}'' etc.)
  from a file name.  \var{Proc} takes a file name as an argument and
  must return a string.  (This is relevant in directory listings.)  The default is a procedure able to handle the
  more common file extensions.
\end{desc}

\defun{with-file-name->content-encoding}{proc [options]}{options}
\begin{desc}
  This specifies a procedure for determining the MIME content encoding
  (if the file is compressed, gzipped, etc.) from a file name.
  (This is relevant in directory listings.)  
  \var{Proc} takes a file name as an argument and must return two
  values: the equivalent, unencoded file name (i.e., without the
  trailing \ex{.Z} or \ex{.gz}) and a string representing the content
  encoding.
\end{desc}

\defun{with-file-name->icon-url}{proc [options]}{options}
\begin{desc}
  This specifies a procedure for determining the icon to be displayed
  next to a file name in a directory listing.
  \var{Proc} takes a file name as an argument and must return a URL
  for the corresponding icon or \sharpf.
\end{desc}

\defun{with-blank-icon-url}{file-name-or-\sharpf{} [options]}{options}
\begin{desc}
  This specifies a file name (or its absence) for the special icon
  that must be as wide as the icons returned by the previous procedure
  but that is blank.
\end{desc}

\defun{with-back-icon-url}{file-name-or-\sharpf{} [options]}{options}
\begin{desc}
  This specifies a file name (or its absence) for the special icon
  that is displayed next to the ``parent directory'' link in directory
  listings.
\end{desc}

\defun{with-unknown-icon-url}{file-name-or-\sharpf{}
  [options]}{options}
\begin{desc}
  This specifies a file name (or its absence) for the special icon
  that is displayed next to the unknown entries in directory listings.
\end{desc}

The \ex{make-file-directory-options} procedure eases the construction
of the options argument:

\defun{make-file-directory-options}{transformer value \ldots}{options}
\begin{desc}
  This constructs an options value from an argument list of parameter
  transformers and parameter values.  The arguments come in pairs,
  each an option transformer from the list above, and a value for that
  parameter.  \ex{Make-file-directory-options} returns the resulting
  options value.
\end{desc}
%
Here are procedure for constructing static content request handlers:
%
\defun{rooted-file-handler}{root [options]}{request-handler} 
\begin{desc}
  This returns a request handler that serves files from a particular
  root in the file system.  Only the \ex{GET} operation is provided.
  The path argument passed to the handler is converted into a
  filename, and appended to \var{root}.  The file name is checked for
  \ex{..}  components, and the transaction is aborted if it does.
  Otherwise, the file is served to the client.
\end{desc}

\defun{rooted-file-or-directory-handler}{root [options]}{request-handler} 
\begin{desc}
Dito, but also serve directory indices for directories without
\ex{index.html}. 
\end{desc}

\defun{home-dir-handler}{subdir [options]}{request-handler}
\begin{desc}
  This procedure builds a request handler that does basic file serving
  out of home directories. If the resulting \var{request-handler} is
  passed a path of the form \ex{(\var{user} . \var{file-path})}, then it serves the file
  \ex{\var{subdir}/\var{file-path}} inside the user's home directory.
    
  The request handler only handles GET requests; the filename is not
  allowed to contain \ex{..} elements.
\end{desc}

\defun{tilde-home-dir-handler}{subdir default-request-handler [options]}{request-handler}
\begin{desc}
  This returns request handler that examines the car of the path. If
  it is a string beginning with a tilde, e.g., \ex{"~ziggy"}, then the
  string is taken to mean a home directory, and the request is served
  similarly to a home-dir-handler request handler. Otherwise, the
  request is passed off in its entirety to the
  \var{default-request-handler}.
\end{desc}
 
\section{CGI Server}

The procedure(s) described here live in the \ex{httpd-cgi-handlers}
structure.

\defun{cgi-handler}{bin-dir [cgi-bin-path]}{request-handler}
\begin{desc}
  Returns a request handler for CGI scripts located in
  \var{bin-dir}.  \var{Cgi-bin-dir} specifies the value of the
  \ex{PATH} variable of the environment the CGI scripts run in. It defaults
  to
\begin{verbatim}
/bin:/usr/bin:/usr/ucb:/usr/bsd:/usr/local/bin
\end{verbatim}
  The CGI scripts are called as specified by CGI/1.1\footnote{see
    \url{http://hoohoo.ncsa.uiuc.edu/cgi/interface.html} for a sort of
    specification.}.

  Note that the CGI handler looks at the name of the CGI script to
  determine how it should be handled:
  \begin{itemize}
  \item If the name of the script starts with `\ex{nph-}', its reply
    is read, the RFC~822-fields like \ex{Content-Type} and \ex{Status}
    are parsed and the client is sent back a real HTTP reply,
    containing the rest of the script's output.
    
  \item If the name of the script doesn't start with `\ex{nph-}',
    its output is sent back to the client directly. If its return code
    is not zero, an error message is generated.
\end{itemize}
\end{desc}      

\section{Scheme-Evaluating Request Handlers}

The \ex{httpd-seval-handlers} structure contains a handler which
demonstrates how to safely evaluate Scheme code uploaded from the
client to the server.

\defvar{seval-handler}{request-handler}
\begin{desc}
  This request handler is suitable for receiving code entered into an
  HTML text form. The Scheme code being uploaded is being \ex{POST}ed
  to the server (from a form).  The code should be URI-encoded in the
  URL as \texttt{program=}$\left<\mathrm{stuff}\right>$.
  $\mathrm{stuff}$ must be an (URI-encoded) Scheme expression which
  the handler evaluates in a separate subprocess.  (It waits for 10
  seconds for a result, then kills the subprocess.)  The handler then
  prints the return values of the Scheme code.
\end{desc}

The following structures define environments that are \RnRS without
features that could examine or effect the file system.  You can also
use them as models of how to execute code in other protected
environments in \scm.

\subsection{The \protect{\texttt{loser}} structure}
The \ex{loser} package exports only one procedure:

\begin{defundesc}{loser}{name}{nothing}
  Raises an error like ``Illegal call \var{name}''.
\end{defundesc}

\subsection{The \protect{\texttt{toothless}} structure}
The \ex{toothless} structure contains everything of \RnRS except
that following procedure cause an error if called:
\begin{itemize}
\item \ex{call-with-input-file}
\item \ex{call-with-output-file}
\item \ex{load}
\item \ex{open-input-file}
\item \ex{open-output-file}
\item \ex{transcript-on}
\item \ex{with-input-from-file}
\item \ex{with-input-to-file}
\item \ex{eval}
\item \ex{interaction-environment}
\item \ex{scheme-report-environment}
\end{itemize}

\subsection{The \protect{\texttt{toothless-eval}} structure}

\begin{defundesc}{eval-safely} {expression} {any result}
  Creates a brand-new structure, imports the \ex{toothless} structure,
  and evaluates \semvar{expression} in it. When the evaluation is
  done, the environment is thrown away, so \semvar{expression}'s
  side-effects don't persist from one \ex{eval\=safely} call to the
  next.  If \semvar{expression} raises an error exception,
  \ex{eval-safely} returns \sharpf.
\end{defundesc}

\section{Writing Request Handlers}

\subsection{Parsing HTML Forms}

In HTML forms, field data are turned into a single string, of the form
\texttt{\synvar{name}=\synvar{val}\&\synvar{name}=\synvar{val}\ldots}.
The \ex{parse-html-forms} structure provides simple functionality to
parse these strings.

\defun{parse-html-form-query}{string}{alist}
\begin{desc}
  This parses \verb|"foo=x&bar=y"| into \verb|(("foo" . "x") ("bar" .
  "y"))|.  Substrings are plus-decoded (i.-e.\ plus characters are
  turned into spaces) and then URI-decoded.

  This implementation is
  slightly sleazy as it will successfully parse a string like
  \verb|"a&b=c&d=f"| into \verb|(("a&b" . "c") ("d" .  "f"))| without
  a complaint.
\end{desc}

\section{SSL encryption with Apache}

Network traffic with a HTTP server is usually encrypted and protected
from manipulation using the cryptographic algorithm provided by an
implementation of the \textit{secure socket layer}, SSL for short.
SUnet does not have support for SSL yet.  However, an Apache
web-server with SSL support can be configured as a proxy.  In this
setup the Apache web-server accepts encrypted requests and forwards
them to a SUnet web-server running locally.  This section describes
how to set up Apache as an encrypting proxy, assuming the reader has
basic knowledge about Apache and its configuration directives.

The following excerpt shows a minimalist SSL virtual host that
forwards requests to a SUnet server.

\begin{alltt}
<VirtualHost 134.2.12.82:443>
  DocumentRoot "/www/some-domain/htdocs"
  ServerName www.some-domain.de
  ServerAdmin admin@some-domain.de
  ErrorLog /www/some-domain/logs/error_log

  ProxyRequests off
  ProxyPass / http://localhost:8080/
  ProxyPassReverse / http://localhost:8080/

  SSLEngine on
  SSLRequireSSL

  SSLCertificateFile /www/some-domain/cert/some-domain.cert
  SSLCertificateKeyFile /www/some-domain/cert/some-domain.key
</VirtualHost>
\end{alltt}

First, a virtual host is added to Apache's configuration file.  This
virtual host listens for incoming connections on port 443, which is
the standard port for encrypted HTTP traffic.  \texttt{SSLRequireSSL}
ensures that server accepts encrypted connections only.

In terms of the Apache documentation, the web-server acts as a so
called \textit{reverse proxy}.  The option \texttt{ProxyRequests} has
a misleading name.  Setting this option to off does only turns off
Apache's facility to act as a \textit{forward proxy} and has no effect
on the configuration directives for reverse proxies.  Actually,
turning on \texttt{ProxyRequests} is dangerous, because this turns
Apache into a proxy server that can be used from anywhere to access
any site that is accessible to the Apache server.

In this setting, all requests get forwarded to a SUnet web-server
which listens for incoming connections on localhost port 8080 only,
thus, it is not reachable from a remote machine.  Apache forwards all
requests to the host and port specified by the \texttt{ProxyPass}
directive.  \texttt{ProxyPassReverse} specifies how
\texttt{Location}-Header fields of HTTP redirect messages send by the
SUNet server are translated.

%%% Local Variables: 
%%% mode: latex
%%% TeX-master: "man"
%%% End: 
