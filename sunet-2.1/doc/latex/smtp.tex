\chapter{SMTP Client}\label{cha:smtp}
%
The \ex{smtp} structure provides an client library for the Simple Mail
Transfer Protocol, commonly used for sending email on the Internet.
This library provides a simple wrapper for sending complete emails as
well as procedures for composing custom SMTP transactions.

Some of the procedures described here return an SMTP reply code.  For
details, see RFC~821.

\defun{smtp-send-mail}{from to-list headers body [host]}{undefined}
\defunx{smtp-error?}{thing}{boolean}
\defunx{smtp-recipients-rejected-error?}{thing}{boolean}
\begin{desc}
  This emails message \var{body} with headers \var{headers} to
  recipients in list \var{to-list}, using a sender address \var{from}.
  The email is handed off to the SMTP server running on \var{host};
  default is the local host.  \var{Body} is either a list of strings
  representing the lines of the message body or an input port which is
  exhausted to determine the message body.  \var{Headers} is an
  association lists, mapping symbols representing RFC~822 field names
  to strings representing field bodies.
  
  If some transaction-related error happens, \ex{smtp-send-mail}
  signals an \ex{smtp-error} condition with predicate
  \ex{smtp-error?}.  More specifically, it raises an
  \ex{smtp-recipients-rejected-error} (a subtype of \ex{smtp-error})
  if some recipients were rejected.  For \ex{smtp-error}, the
  arguments to the \ex{signal} call are the error code and the error
  message, represented as a list of lines.  For
  \ex{smtp-recipients-rejected-error}, the arguments are reply code
  700 and an association list whose elements are of the form
  \ex{(\var{loser-recipient} \var{code} .  \var{text})}---that is, for
  each recipient refused by the server, you get the error data sent
  back for that guy.  The success check is \ex{(< code 400)}.
\end{desc}

\defun{smtp-expand}{name host}{code text}
\defunx{smtp-verify}{name host}{code text}
\defunx{smtp-get-help}{host [details]}{code text-list}
\begin{desc}
  These three are simple queries of the server as stated in the
  RFC~821: \ex{smtp-expann} asks the server to confirm that the
  argument identifies a mailing list, and if so, to return the
  membership of that list.  The full name of the users (if known) and
  the fully specified mailboxes are returned in a multiline reply.
  \ex{Smtp-verify} asks the receiver to confirm that the argument
  identifies a user.  If it is a user name, the full name of the user
  (if known) and the fully specified mailbox are returned.
  \ex{Smtp-get-help} causes the server to send helpful information.
  The command may take an argument (\var{details}) (e.g., any command
  name) and return more specific information as a response.
\end{desc}

\defun{smtp-connect}{host [port]}{smtp-connection}
\begin{desc}
  \ex{Smtp-connect} returns an SMTP connection value that represents
  a connection to the SMTP server.
\end{desc}

\defun{smtp-transactions}{smtp-connection transaction1 ...}{code text-list}
\defunx{smtp-transactions/no-close}{smtp-connection transaction1 ...}{code text-list}
\begin{desc}
  These procedures make it easy to do simple sequences of SMTP
  commands.  \var{Smtp-connection} must be an SMTP connection as
  returned by \ex{smtp-connect}.  The \var{transaction} arguments must
  be transactions as returned by the procedures below.
  \ex{Smtp-transactions} and \ex{smtp-transactions/no-close} execute
  the transactions specified by the arguments.

  For each transaction,
  \begin{itemize}
  \item If the transaction's reply code is 221 or 421 (meaning the socket has
    been closed), then the transaction sequence is aborted, and 
    \ex{smtp-transactions}/\ex{smtp-transactions/no-close} return the
    reply code and text from that  transaction.
  \item If the reply code is an error code (in the four- or five-hundred range), 
    the transaction sequence is aborted, and the fatal transaction's code
    and text values are returned.  \ex{Smtp-transactions} will additionally
    close the socket for you; \ex{smtp-transactions/no-close} will not.
  \item  If the transaction is the last in the transaction sequence, 
    its reply code and text are returned.
  \item Otherwise, we throw away the current reply code and text, and
    proceed to the next transaction.
  \end{itemize}
                                %
  \ex{Smtp-transactions} closes the socket after the transaction.  (The
  \ex{smtp-quit} transaction, when executed, also closes the transaction.)
  
  If the socket should be kept open in the case of an abort, use
  \ex{Smtp-transactions/no-close}.
\end{desc}

\defunx{smtp-helo}{local-host-name}{smtp-transaction}
\defunx{smtp-mail}{sender-address}{smtp-transaction}
\defunx{smtp-rcpt}{destination-address}{smtp-transaction}
\defunx{smtp-data}{socket message}{smtp-transaction}
\defunx{smtp-send}{sender-address}{smtp-transaction}
\defunx{smtp-soml}{sender-address}{smtp-transaction}
\defunx{smtp-saml}{sender-address}{smtp-transaction}
\defvarx{smtp-rset}{smtp-transaction}
\defunx{smtp-vrfy}{user}{smtp-transaction}
\defunx{smtp-expn}{user}{smtp-transaction}
\defunx{smtp-help}{details}{smtp-transaction}
\defvarx{smtp-noop}{smtp-transaction}
\defvarx{smtp-quit}{smtp-transaction}
\defvarx{smtp-turn}{smtp-transaction}
\begin{desc}
  These transactions represent the commands of the SMTP protocol for
  use in \ex{smtp-transactions} and \ex{smtp-transactions/no-close},
  i.e.\ they send the corresponding command along with the argument(s),
  if any.  For details, consult RFC~821.
  
  The \ex{smtp-quit} transaction, in addition to sending a \ex{QUIT}
  command to the SMTP server, also closes the socket of its SMTP
  connection.
\end{desc}


%%% Local Variables: 
%%% mode: latex
%%% TeX-master: "man"
%%% End: 
