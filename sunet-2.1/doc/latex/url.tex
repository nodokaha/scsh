\chapter{Parsing and Processing URLs}\label{cha:url}
%
This modules contains procedures to parse and unparse URLs.  Until
now, only the parsing of HTTP URLs is implemented.

\section{Server Records}

A \textit{server} value describes path prefixes of the form
\var{user}:\var{password}@\var{host}:\var{port}. These are
frequently used as the initial prefix of URLs describing Internet
resources.

\defun{make-server}{user password host port}{server}
\defunx{server?}{thing}{boolean}
\defunx{server-user}{server}{string-or-\sharpf}
\defunx{server-password}{server}{string-or-\sharpf}
\defunx{server-host}{server}{string-or-\sharpf}
\defunx{server-port}{server}{string-or-\sharpf}
\begin{desc}
  \ex{Make-server} creates a new server record.  Each slot is a
  decoded string or \sharpf. (\var{Port} is also a string.)
  
  \ex{server?} is the corresponding predicate, \ex{server-user},
  \ex{server-password}, \ex{server-host} and \ex{server-port}
  are the correspondig selectors.
\end{desc}

\defun{parse-server}{path default}{server}
\defunx{server->string}{server}{string}
\begin{desc}
  \ex{Parse-server} parses a URI path \var{path} (a list representing
  a path, not a string) into a server value.  Default values are taken
  from the server \var{default} except for the host.  The values
  are unescaped and stored into a server record that is returned.
  \ex{Fatal-syntax-error} is called, if the specified path has no
  initial to slashes (i.e., it starts with `//\ldots').
  
  \ex{server->string} just does the inverse job: it unparses
  \var{server} into a string. The elements of the record
  are escaped before they are put together.  

  Example:
\begin{alltt}
> (define default (make-server "andreas"  "se ret" "www.sf.net" "80"))
> (server->string default)
"andreas:se\%20ret@www.sf.net:80"
> (parse-server '("" "" "foo\%20bar@www.scsh.net" "docu" "index.html") 
                default)
'#{server}
> (server->string ##)
"foo\%20bar:se\%20ret@www.scsh.net:80"
\end{alltt}
%
For details about escaping and unescaping see Chapter~\ref{cha:uri}.
\end{desc}

\section{HTTP URLs}

\defun{make-http-url}{server path search frag-id}{http-url}
\defunx{http-url?}{thing}{boolean}
\defunx{http-url-server}{http-url}{server}
\defunx{http-url-path}{http-url}{list}
\defunx{http-url-search}{http-url}{string-or-\sharpf}
\defunx{http-url-frag-ment-identifier}{http-url}{string-or-\sharpf}
%
\begin{desc}
  \ex{Make-http-url} creates a new \ex{httpd-url} record.
  \var{Server} is a record, containing the initial part of the address
  (like \ex{anonymous@clark.lcs.mit.edu:80}).  \var{Path} contains the
  URL's URI path ( a list).  These elements are in raw, unescaped
  format. To convert them back to a string, use
  \ex{(uri-path->uri (map escape-uri pathlist))}. \var{Search}
  and \var{frag-id} are the last two parts of the URL.  (See
  Chapter~\ref{cha:uri} about parts of an URI.)
  
  \ex{Http-url?} is the predicate for HTTP URL values, and
  \ex{http-url-server}, \ex{http-url-path}, \ex{http-url-search} and
  \ex{http-url-fragment-identifier} are the corresponding selectors.
\end{desc}

\defun{parse-http-url}{path search frag-id}{http-url}
\begin{defundescx}{http-url->string}{http-url}{string}
  This constructs an HTTP URL record from a URI path (a list of path
  components), a search, and a frag-id component.
  
  \ex{Http-url->string} just does the inverse job. It converts an
  HTTP URL record into a string.
\end{defundescx}
%
Note: The URI parser \ex{parse-uri} maps a string to four parts:
\var{scheme}, \var{path}, \var{search} and \var{frag-id} (see
Section~\ref{proc:parse-uri} for details). If \var{scheme} is
\ex{http}, then the other three parts can be passed to
\ex{parse-http-url}, which parses them into a \ex{http-url} record.
All strings come back from the URI parser encoded.  \var{Search} and
\var{frag-id} are left that way; this parser decodes the path
elements.  The first two list elements of the path indicating the
leading double-slash are omitted.

The following procedure combines the jobs of \ex{parse-uri} and
\ex{parse-http-url}:

\defun{parse-http-url-string}{string}{http-url}
\begin{desc}
  This parses an HTTP URL and returns the corresponding URL value; it
  calls \ex{fatal-syntax-error} if the URL string doesn't have an
  \ex{http} scheme.
\end{desc}

%%% Local Variables: 
%%% mode: latex
%%% TeX-master: "man"
%%% End: 
