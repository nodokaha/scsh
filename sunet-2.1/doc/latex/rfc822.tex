\chapter{RFC~822 Library}\label{cha:rfc822}
%
The \ex{rfc822} structure provides rudimentary support for parsing
headers according to RFC~822 \textit{Standard for the format of ARPA
  Internet text messages}.  These headers show up in SMTP messages,
HTTP headers, etc.

An RFC~822 header field consists of a \textit{field name} and a
\textit{field body}, like so:
%
\begin{verbatim}
Subject: RFC 822 can format itself in the ARPA
\end{verbatim}
%
Here, the field name is `\ex{Subject}', and the field name is `\ex{
  RFC 822 can format itself in the ARPA}' (note the leading space).
The field body can be spread over several lines:
%
\begin{verbatim}
Subject: RFC 822 can format itself
  in the ARPA
\end{verbatim}
%
In this case, RFC~822 specifies that the meaning of the field body is
actually all the lines of the body concatenated, without the
intervening line breaks.

The \ex{rfc822} structure provides two sets of parsing
procedures---one represents field bodies in the RFC-822-specified
meaning, as a single string, the other (with \ex{-with-line-breaks}
appended to the names) reflects the line breaks and represents the
bodies as a list of string, one for each line.  The latter set only
marginally useful---mainly for code that needs to output headers in
the same form as they were originally provided.

\defun{read-rfc822-field}{[port] [read-line]}{name body}
\defun{read-rfc822-field-with-line-breaks}{[port] [read-line]}{name body-lines}
\begin{desc}
  
  Read one field from the port, and return two values:
  %
  \begin{description}
  \item[\var{name}] This is a symbol describing the field
    name, such as \ex{subject} or \ex{to}.  The symbol consists of all
    lower-case letters.\footnote{In fact, it \ex{read-rfc822-field}
      uses the preferred case for symbols of the underlying Scheme
      implementation which, in the case of scsh, happens to be lower-case.}
  \item[\var{body} or \var{body-lines}] This is the field body.
    \var{Body} is a single string, \var{body-lines} is a list of
    strings, one for each line of the body.  In each case,
    the terminating \ex{cr}/\ex{lf}'s (but nothing else) are
    trimmed from each string.
  \end{description}
  %
  When there are no more fields---EOF or a blank line has terminated
  the header section---then both procedures returns [\sharpf\ 
  \sharpf].

  \var{Port} is an optional input port to read from---it defaults to
  the value of \ex{(current-input-port)}.
  
  \var{Read-line} is an optional parameter specifying a procedure of
  one argument (the input port) used to read the raw header lines.
  The default used by these procedures terminates lines with
  either \ex{cr}/\ex{lf} or just \ex{lf}, and it trims the terminator
  from the line.  This procedure should trim the terminator of the
  line, so an empty line is returned as an empty string.
  
  The procedure raises an error if the syntax of the read field (the
  line returned by the read-line-function) is illegal according to
  RFC~822.
\end{desc}

\defun{read-rfc822-headers} {[port] [read-line]} {alist}
\defunx{read-rfc822-headers-with-line-breaks} {[port] [read-line]} {alist}
\begin{desc}
  This procedure reads in and parses a section of text that looks like
  the header portion of an RFC~822 message.  It returns an association
  list mapping field names (a symbol such as \ex{date} or \ex{subject}) to 
  field bodies.  The representation of the field bodies is as with
  \ex{read-rfc822-field} and \ex{read-rfc822-field-with-line-breaks}.
  
  These procedures preserve the order of the header fields.  Note that
  several header fields might share the same field name---in that
  case, the returned alist will contain several entries with the same
  \ex{car}.
  
  \var{Port} and \var{read-line} are as with \ex{read-rfc822-field}
  and \ex{read-rfc822-field-with-line-breaks}.
\end{desc}

\defun{rfc822-time->string}{time}{string}
\begin{desc}
  This formats a time value (as returned by scsh's \ex{time})
  according to the requirements of the RFC~822 \ex{Date} header
  field.  The format looks like this:
%
\begin{verbatim}
Sun, 06 Nov 1994 08:49:37 GMT
\end{verbatim}
\end{desc}


%%% Local Variables: 
%%% mode: latex
%%% TeX-master: "man"
%%% End: 
