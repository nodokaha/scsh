\chapter{Parsing Netrc Files}\label{cha:netrc}
%
The \ex{netrc} structures provides procedures to parse authentication
information contained in \ex{~/.netrc}.

On Unix systems the netrc file may contain information allowing
automatic login to remote hosts.  The format of the file is defined in
the \ex{ftp(1)} manual page.  Example lines are
%
\begin{verbatim}
machine ondine.cict.fr login marsden password secret
default login anonymous password user@site
\end{verbatim}
%
The netrc file should be protected by appropriate permissions, and
(like \ex{/usr/bin/ftp}) this library will refuse to read the file if it is
badly protected.  (unlike \ex{ftp} this library will always refuse
to read the file----\ex{ftp} refuses it only if the password is
given for a non-default account).  Appropriate permissions are set if
only the user has permissions on the file.

\defun{netrc-machine-entry}{host accept-default? [file-name]}{netrc-entry-or-\sharpf}
\begin{desc}
  This procedure looks for the entry related to given host in the
  user's netrc file.  The host is specified in \var{host}.
  \var{Accept-default?} specifies whether \ex{netrc-machine-entry}
  should fall back to the default entry if there is no macht for
  \var{host} in the netrc file.  If specified, \var{file-name}
  specifies an alternate file name for the netrc data.  It defaults to
  \ex{.netrc} in the current user's home directory.

  \ex{Netrc-machine-entry} returns a netrc entry (see below) if it was
  able to find the requested information; if not, it returns \sharpf.

  If the netrc file had inappropriate permissions, \ex{netrc-machine-entry}
  raises an error.
\end{desc}

\defun{netrc-entry?}{thing}{boolean}
\defunx{netrc-entry-machine}{netrc-entry}{string}
\defunx{netrc-entry-login}{netrc-entry}{string-or-\sharpf}
\defunx{netrc-entry-password}{netrc-entry}{string-or-\sharpf}
\defunx{netrc-entry-account}{netrc-entry}{string-or-\sharpf}
\begin{desc}
  \ex{Netrc-entry?} is the predicate for netrc entries.  The other
  procedures are selectors for netrc entries as returned by
  \ex{netrc-machine-entry}.  They return \sharpf{} if the netrc file
  didn't contain a binding for the corresponding field.
\end{desc}

\defun{netrc-macro-definitions}{[file-name]}{alist}
\begin{desc}
  This returns the macro definitions from the netrc files, represented
  as an alist mapping macro names---represented as strings---to
  definitions---represented as lists of strings.
\end{desc}

%%% Local Variables: 
%%% mode: latex
%%% TeX-master: "man"
%%% End: 
