\chapter{Parsing and Processing URIs}\label{cha:uri}

The \ex{uri} structure contains a library for dealing with URIs.

\section{Notes on URI Syntax}

A URI (Uniform Resource Identifier) is of following syntax:
%
\begin{inset}
[\var{scheme}] \verb|:| \var{path} [\verb|?| \var{search}] [\verb|#| \var{fragid}]
\end{inset}
%
Parts in brackets may be omitted.

The URI contains characters like \verb|:| to indicate its different
parts.  Some special characters are \emph{escaped} if they are a
regular part of a name and not indicators for the structure of a URI.
Escape sequences are of following scheme: \verb|%|\var{h}\var{h} where \var{h}
is a hexadecimal digit.  The hexadecimal number refers to the
ASCII of the escaped character, e.g.\ \verb|%20| is space (ASCII
32) and \verb|%61| is `a' (ASCII 97). This module
provides procedures to escape and unescape strings that are meant to
be used in a URI.

\section{Procedures}

\defun{parse-uri} {uri-string } {scheme path search
    frag-id} \label{proc:parse-uri}
\begin{desc}
  Parses an \var{uri\=string} into its four fields.
  The fields are \emph{not} unescaped, as the rules for
  parsing the \var{path} component in particular need unescaped
  text, and are dependent on \var{scheme}. The URL parser is
  responsible for doing this.  If the \var{scheme}, \var{search}
  or \var{fragid} portions are not specified, they are \sharpf.
  Otherwise, \var{scheme}, \var{search}, and \var{fragid} are
  strings. \var{path} is a non-empty string list---the path split
  at slashes.
\end{desc}

Here is a description of the parsing technique. It is inwards from
both ends:
\begin{itemize}
\item First, the code searches forwards for the first reserved
  character (\verb|=|, \verb|;|, \verb|/|, \verb|#|, \verb|?|,
  \verb|:| or \verb|space|).  If it's a colon, then that's the
  \var{scheme} part, otherwise there is no \var{scheme} part. At
  all events, it is removed.
\item Then the code searches backwards from the end for the last reserved
  char.  If it's a sharp, then that's the \var{fragid} part---remove it.
\item Then the code searches backwards from the end for the last reserved
  char.  If it's a question-mark, then that's the \var{search}
  part----remove it.
\item What's left is the path.  The code split it at slashes. The
  empty string becomes a list containing the empty string.
\end{itemize}
%  
This scheme is tolerant of the various ways people build broken
URI's out there on the Net\footnote{So it does not absolutely conform
  to RFC~1630.}, e.g.\ \verb|=| is a reserved character, but used
unescaped in the search-part. It was given to me\footnote{That's
  Olin Shivers.} by Dan Connolly of the W3C and slightly modified.

\defun{unescape-uri}{string [start] [end]}{string}
\begin{desc}
  \ex{Unescape-uri} unescapes a string. If \var{start} and/or \var{end} are
  specified, they specify start and end positions within \var{string}
  should be unescaped.
\end{desc}
%
This procedure should only be used \emph{after} the URI was parsed,
since unescaping may introduce characters that blow up the
parse---that's why escape sequences are used in URIs.

\defvar{uri-escaped-chars}{char-set}
\begin{desc}
  This is a set of characters (in the sense of SRFI~14) which are
  escaped in URIs.  RFC 2396 defines this set as all characters which 
  are neither letters, nor digits, nor one of the following characters:
   \verb|-|, \verb|_|, \verb|.|, \verb|!|, %$
   \verb|~|, \verb|*|, \verb|'|, \verb|(|, \verb|)|.
\end{desc}

\defun{escape-uri} {string [escaped-chars]} {string}
\begin{desc}
  This procedure escapes characters of \var{string} that are in
  \var{escaped\=chars}. \var{Escaped\=chars} defaults to
  \ex{uri\=escaped\=chars}.  
\end{desc}
%
Be careful with using this procedure to chunks of text with
syntactically meaningful reserved characters (e.g., paths with URI
slashes or colons)---they'll be escaped, and lose their special
meaning. E.g.\ it would be a mistake to apply \ex{escape-uri} to
\begin{verbatim}
//lcs.mit.edu:8001/foo/bar.html
\end{verbatim}
%
because the sla\-shes and co\-lons would be escaped.

\defun{split-uri}{uri start end} {list}
\begin{desc}
  This procedure splits \var{uri} at slashes. Only the substring given
  with \var{start} (inclusive) and \var{end} (exclusive) as indices is
  considered.  \var{start} and $\var{end} - 1$ have to be within the
  range of \var{uri}.  Otherwise an \ex{index-out-of-range} exception
  will be raised.
  
  Example: \codex{(split-uri "foo/bar/colon" 4 11)} returns
  \codex{("bar" "col")}
\end{desc}

\defun{uri-path->uri}{path}{string}
\begin{desc}
  This procedure generates a path out of a URI path list by inserting
  slashes between the elements of \var{plist}.
\end{desc}
%
If you want to use the resulting string for further operation, you
should escape the elements of \var{plist} in case they contain
slashes, like so:
%
\begin{verbatim}
(uri-path->uri (map escape-uri pathlist))
\end{verbatim}

\defun{simplify-uri-path}{path}{list}
\begin{desc}
  This procedure simplifies a URI path.  It removes \verb|"."| and
  \verb|"/.."| entries from path, and removes parts before a root.
  The result is a list, or \sharpf{} if the path tries to back up past
  root.
\end{desc}
%
According to RFC~2396, relative paths are considered not to start with
\verb|/|.  They are appended to a base URL path and then simplified.
So before you start to simplify a URL try to find out if it is a
relative path (i.e. it does not start with a \verb|/|).

Examples:
%
\begin{alltt}
(simplify-uri-path (split-uri  "/foo/bar/baz/.."  0 15))
\(\Rightarrow\) ("" "foo" "bar")

(simplify-uri-path (split-uri "foo/bar/baz/../../.." 0 20))
\(\Rightarrow\) ()

(simplify-uri-path (split-uri "/foo/../.." 0 10))
\(\Rightarrow\) #f

(simplify-uri-path (split-uri "foo/bar//" 0 9))
\(\Rightarrow\) ("")     

(simplify-uri-path (split-uri "foo/bar/" 0 8))
\(\Rightarrow\) ("")

(simplify-uri-path (split-uri "/foo/bar//baz/../.." 0 19))
\(\Rightarrow\) #f
\end{alltt}


%%% Local Variables: 
%%% mode: latex
%%% TeX-master: "man"
%%% End: 
