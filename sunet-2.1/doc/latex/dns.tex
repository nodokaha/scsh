\chapter{DNS Client Library}\label{cha:dns}
%
\begin{description}
\item[Used files:] dns.scm
\item[Name of the package:] dns
\end{description}
%
\section{Overview}
The \ex{dns} structure contains a library for querying DNS servers.
The library contains sophisticated replacements for scsh's interface
to the \ex{gethostbyname} and \ex{gethostbyaddr} and many extensions
to these functions.

The main features of the libraray include:
\begin{itemize}
\item Complete implementation of the DNS protocol
\item Concurrent contacting of multiple DNS servers without blocking
  the scsh process
\item Internal caching of DNS responses
\item Parsing of \texttt{resolv.conf}, including \texttt{search}
  entries to generate FQDNs from unqualified host names
\item Rich condition hierarchie
\end{itemize}

\section{Conditions}

The library defines a set of conditions raised by the procedures of
the library. The supertype of these conditions is \exi{dns-error}.
\defun{dns-error?}{thing}{\boolean}
\begin{desc}
  The predicate for \ex{dns-error} conditions.
\end{desc}
\defun{dns-error->string} {dns-error-condition} {\str}
\begin{desc}
  Returns a string with the description of the condition.
\end{desc}

\defvar{parse-error}{condition}
\defvarx{unexpected-eof-from-server}{condition}
\defvarx{bad-address}{condition}
\defvarx{no-nameservers}{condition}
\defvarx{bad-nameserver}{condition}
\defvarx{not-a-hostname}{condition}
\defvarx{not-a-ip} {condition}

  \begin{desc}
    
  \end{desc}
\defvar {dns-format-error} {condition}
\defvarx {dns-server-failure} {condition}
\defvarx {dns-name-error} {condition}
\defvarx {dns-not-implemented} {condition}
\defvarx {dns-refused} {condition}
\begin{desc}
  These conditons correspond to errors returned by the DNS server.
  They are all subtypes of the \exi{dns-server-error} condition which
  in turn is a subtype of \ex{dns-error}.
\end{desc}
\defun{dns-server-error?}{thing}{\boolean}
\begin{desc}
  The predicate for \ex{dns-server-error} conditions.
\end{desc}

\defun{parse-error?}{thing} {\boolean}
\defunx{unexpected-eof-from-server?}{thing} {\boolean}
\defunx{bad-address?}{thing} {\boolean}
\defunx{no-nameservers?}{thing} {\boolean}
\defunx{bad-nameserver?}{thing} {\boolean}
\defunx{not-a-hostname?}{thing} {\boolean}
\defunx{not-a-ip?}{thing} {\boolean}
\defunx{dns-format-error?} {thing} {\boolean}
\defunx{dns-server-failure?} {thing} {\boolean}
\defunx{dns-name-error?} {thing} {\boolean}
\defunx{dns-not-implemented?} {thing} {\boolean}
\defunx{dns-refused?} {thing} {\boolean}
\begin{desc}
  The type predicates for the conditions above.
\end{desc}

\section{High-level Interface}
\def\ipaddr{IP-address\xspace}
\def\ipstring{IP-string\xspace}
\def\fqdn{FQDN\xspace}

The library uses an internal store to cache data obtained from DNS
servers. All procedures take a boolean flag \var{use-cache?} that
indicates whether the cache should be used or not. \var{use-cache?}
defaults to true.

\defun{dns-clear-cache!}{}{\undefined}
\begin{desc}
  This procedure erases all information stored in the internal cache.
\end{desc}

The library is further capable of parsing the contents of
\texttt{/etc/resolv.conf} (see Section~\ref{sec:dns-rc}). The
nameservers listed there are the default value for the optional
argument \var{nameserver list} which many procedures of the library
accept. \var{Nameserver} is either a \ipaddr or a dotted IP string.

\defun{dns-lookup-name}{\fqdn [nameserver list][use-cache?]}{\ipaddr}
\begin{desc}
  Given the FQDN of a host, \ex{dns-lookup-ip} returns the IP address.
  The optional argument specifes the name servers to query, it defaults
  to the ones found in \texttt{/etc/resolv.conf}. 
\end{desc}

\defun{dns-lookup-ip}{\ipstring/\ipaddr [nameserver list][use-cache?]}{\fqdn}
\begin{desc}
  Looks up the FQDN for the given IP address. The optional argument
  specifes the name servers to query, it defaults to the ones found in
  \texttt{/etc/resolv.conf}.
\end{desc}

\defun{dns-lookup-nameserver}{\ipstring/\ipaddr [nameserver list][use-cache?]}{\ipaddr list}
\begin{desc}
  Looks up an authoritative name server for a hostname, returns a list
  of name servers.
\end{desc}

\defun{dns-lookup-mail-exchanger}{\ipstring/\ipaddr [nameserver list][use-cache?]}{\fqdn list}
\begin{desc}
  Looks up mail-exchangers for a hostname und returns them in a list
  sorted by preference.
\end{desc}
\defun{socket-address->fqdn}{socket-address [nameserver list][use-cache?]}{\fqdn}
\begin{desc}
  Returns the FQDN for of the address bound to argument. The argument
  \var{cache?}  indicates whether the internal cache may be queried to
  obtain the information.
\end{desc}

\defun{maybe-dns-lookup-name}{\fqdn [nameserver list][use-cache?]}{\ipaddr or \sharpf}
\defunx{maybe-dns-lookup-ip}{\ipstring/\ipaddr [nameserver list][use-cache?]}{\fqdn{} or \sharpf}
\begin{desc}
  These procedures provide the same functionality as
  \ex{dns-lookup-name} and \ex{dns-lookup-ip} but return \sharpf{} in
  case of an \ex{dns-error}.
\end{desc}

\defun{host-fqdn} {name/socket-address [nameserver list][use-cache?]}{\fqdn}
\defunx{system-fqdn}{[nameserver list][use-cache?]}{\fqdn}
\begin{desc}
  \ex{host-fqdn} returns the fully qualified domain name (FQDN) for
  its argument which can be either a unqualified host name or a socket
  address.  The procedure \ex{system-fqdn} returns the FQDN of the
  local host. These procedures use a list of domain names obtained
  from \texttt{/etc/resolv.conf} to the generate FQDNs and try to
  resolve these FQDNs.
\end{desc}

\defun{dns-check-nameservers} {[nameserver list]} {\undefined}
\begin{desc}
  \ex{dns-check-namservers} checks if the given nameservers are reachable.
  If no argument is given, the nameservers in \texttt{/etc/resolv.conf}
  are checked. Information about the status of the nameservers is printed
  to the current output port.
\end{desc}

\section{Low-level Interface}

This section describes a set of data structures and procedures which
directly correspond to the data flow of the DNS protocol. The central
entity is a \var{message}, the abstraction of the packet sent to the
server or received from the server (The DNS protocol uses the same
data format for both directions). A \var{dns-message} encapsulates the
query message sent to the server, the response message received from
the server, and some additional information the library gathered while
generating the \var{dns-message}.

\defunx{dns-get-information}{message protocol answer-okay? [nameserver
  list][use-cache?]}{dns-message}
\begin{desc}
  Most general way to submit a DNS query. The message is sent to the
  name servers via \var{protocol} which can be either
  \ex{(network-procotcol tcp)} or {(network-protocol udp)}, both
  members of of the enumerated type \ex{network-protocol}. After
  receiving the reply, \ex{dns-get-information} applies the predicate
  \var{answer-okay?} to the message. If it returns \sharpf{} and the
  answer is not authoritative additional name servers sent with the
  reply are checked until an authoritative answer is found. If the
  predicate returns \sharpf{} but the answer is authoritative a
  \var{bad-address} condition is signalled.
\end{desc}

\dfn{network-protocol}{protocol-name}{network-protocol}{syntax}
\defunx{network-protocol?}{thing}{\boolean}
\begin{desc}
  Constructor and predicate for the enumerated type
  \ex{network-protocol} with the possible protocol names \ex{tcp} and
  \ex{udp}.
\end{desc}
\defun{dns-lookup}{\ipstring/\ipaddr type [nameserver list][use-cache?]}{dns-message}
\begin{desc}
  Convenient shortcut to submit a DNS query. The return value
  is a \ex{dns-message} structure:
\end{desc}

\defun{dns-message?}{thing}{\boolean}
\defunx{dns-message-query}{dns-message}{message}
\defunx{dns-message-reply}{dns-message}{message}
\defunx{dns-message-cache?}{dns-message}{\boolean}
\defunx{dns-message-protocol}{dns-message}{protocol}
\defunx{dns-message-tried-nameservers}{dns-message}{}
\begin{desc}
  A \var{dns-message} records the query sent to the server and the
  reply from the server. It also contains information whether the
  library took the reply from the cache, which protocol was used and
  to which nameservers the query was sent.
\end{desc}

\defun{pretty-print-dns-message}{dns-message [output-port]}{\undefined}
\begin{desc}
  Pretty prints a DNS message to \var{out-port} which defaults to the
  current output port.
\end{desc}

\defun{message?}{thing}{\boolean}{}
\defunx{message-header}{message}{header}
\defunx{message-questions}{message}{question list}
\defunx{message-answers}{message}{resource-record list}
\defunx{message-nameservers}{message}{resource-record list}
\defunx{message-additionals}{message}{resource-record list}
\defunx{message-source}{message}{char list}
\begin{desc}
  A \ex{message} represents the data sent to the DNS server or
  received from the DNS server. The DNS protocol uses the same message
  format for queries and replies. In queries only the header and the
  questions is present, a reply may contain answers, name servers and
  and additional informations as resource records. \ex{Message-source}
  returns the actual data sent over the network.
\end{desc}

\defun{make-query-message header}{header question [questions]}{message}
\begin{desc}
  The procedure generates a message the supplied questions,
  \var{header}, and the standard message values for queries.
\end{desc}
\defun{make-simple-query-message}{name type class}{message}
\begin{desc}
  This simplified constructor generates a message with one question
  which is built from the parameters, and the standard header flags
  for queries and the standard message values for queries.
\end{desc}

\defun{header?}{thing}{\boolean}
\defunx{header-id}{header}{number}
\defunx{header-flags}{header}{flags}
\defunx{header-question-count}{header}{number}
\defunx{header-answer-count}{header}{number}
\defunx{header-nameserver-count}{header}{number}
\defunx{header-additional-count}{header}{number}
\begin{desc}
  Every DNS message contains a header which stores information about
  the data present in the message and contains flags for the query.
\end{desc}

\defun{flags?}{thing}{\boolean}
\defunx{flags-query-type}{flags}{'query or 'response}
\defunx{flags-opcode}{flags}{number}
\defunx{flags-authoritative?}{flags}{\boolean}
\defunx{flags-truncated?}{flags}{\boolean}
\defunx{flags-recursion-desired?}{flags}{\boolean}
\defunx{flags-recursion-available?}{flags}{\boolean}
\defunx{flags-zero}{flags}{0}
\defunx{flags-response-code}{flags}{number}
\begin{desc}
  Flags occur within the header of a DNS message. The boolean value
  returned from \ex{flags-authoritative} indicates whether the message
  was sent from a authoritative server, \ex{flags-truncated?} should
  always be \sharpf as the library automatically uses the TCP protocol
  is the UDP message size is not sufficied.
\end{desc}

\defun{question?}{thing}{\boolean}
\defunx{question-name}{question}{\str}
\defunx{question-type}{question}{message-type}
\defunx{question-class}{question}{message-class}
\begin{desc}
  A question sent to the DNS server.
\end{desc}
The type and class of the question and answer are elements of
enumerated types:
\dfn{message-class}{class-name}{message-class}{syntax}
\defunx{message-class?}{thing}{\boolean}
\defunx{message-class-name}{message-class}{symbol}
\defunx{message-class-number}{message-class}{number}
\begin{desc}
  \ex{message-class} constructs a member of the enumerated type,
  \ex{message-class?} is the type predicate, \ex{message-class-name}
  returns the symbol and \ex{message-class-number} the number used for
  the class in the DNS protocol.
\end{desc}
The possible names for the classes are:
\begin{description}
\item[\ex{in}] The Internet
\item[\ex{cs}] obsolete
\item[\ex{ch}] the CHAOS class
\item[\ex{hs}] Hesoid
\end{description}

\dfn{message-type}{type-name}{message-type}{syntax}
\defunx{message-type?}{thing}{\boolean}
\defunx{message-type-name}{message-type}{symbol}
\defunx{message-type-number}{message-type}{number}
\begin{desc}
  \ex{message-type} constructs a member of the enumeration from name
  \synvar{type-name} listed in Table~\ref{tab:message-types}.
  \ex{message-type?} is the type predicate, \ex{message-type-name}
  returns the name, and \ex{message-type-number} the number used for
  the class the DNS protocol.
  
\end{desc}
\begin{table}[htb]
  \centering
  \begin{tabular}{|l|l|}
    \hline
    \ex{a}& a host address\\\hline
    \ex{ns}&an authoritative name server\\\hline
    \ex{md}&(obsolete)\\\hline
    \ex{mf}&(obsolete)\\\hline
    \ex{cname}&the canonical name for an alias\\\hline
    \ex{soa}& marks the start of a zone of authority\\\hline
    \ex{mb}&(experimental)\\\hline
    \ex{mg}&(experimental)\\\hline
    \ex{mr}&(experimental)\\\hline
    \ex{null}& (experimental)\\\hline
    \ex{wks}& a well known service description\\\hline
    \ex{ptr}& a domain name pointer\\\hline
    \ex{hinfo}& host information\\\hline
    \ex{minfo}& (experimental)\\\hline
    \ex{mx}& mail exchange\\\hline
    \ex{txt}& text strings\\\hline
  \end{tabular}
  \caption{Message types}
  \label{tab:message-types}
\end{table}

\defun{resource-record?}{thing}{\boolean}
\defunx{resource-record-name}{resource-record}{\str}
\defunx{resource-record-type}{resource-record}{message-type}
\defunx{resource-record-class}{resource-record}{message-class}
\defunx{resource-record-ttl}{resource-record}{number}
\defunx{resource-record-data}{resource-record}{resource-record-data-\dots}
\begin{desc}
  A resource record as returned from the DNS server. The actual data
  of the record is stored in the \texttt{resource-record-data} field.
  It is one of the record types for resource record data described
  below.
\end{desc}

\defun{resource-record-data-a?}{thing}{\boolean}
\defunx{resource-record-data-a-ip}{resource-record-data-a}{\ipaddr}
\begin{desc}
  An address resource record which holds an internet address.
\end{desc}

\defun{resource-record-data-ns?}{thing}{\boolean}
\defunx{resource-record-data-ns-name}{resource-record-data-ns}{\fqdn}
\begin{desc}
  A name server resource record containing the FQDN of the name server.
\end{desc}

\defun{resource-record-data-cname?}{thing}{\boolean}
\defunx{resource-record-data-cname-name}{resource-record-data-cname}{\fqdn}
\begin{desc}
  A canonical name resource record which contains the canonical or
  primary name of the owner.
\end{desc}

\defun{resource-record-data-mx?}{thing}{\boolean}
\defunx{resource-record-data-mx-preference}{resource-record-data-mx}{number}
\defunx{resource-record-data-mx-exchanger}{resource-record-data-mx}{\fqdn}
\begin{desc}
  A mail exchange resource record with the preference and the FQDN of
  a host willing to act as a mail exchange.
\end{desc}

\defun{resource-record-data-ptr?}{thing}{\boolean}
\defunx{resource-record-data-ptr-name}{resource-record-data-ptr}{\str}
\begin{desc}
  A pointer resource record which points to some other domain name.
\end{desc}

\defun{resource-record-data-soa?}{thing}{\boolean}
\defunx{resource-record-data-soa-mname}{resource-record-data-soa}{\fqdn}
\defunx{resource-record-data-soa-rname}{resource-record-data-soa}{\fqdn}
\defunx{resource-record-data-soa-serial}{resource-record-data-soa}{number}
\defunx{resource-record-data-soa-refresh}{resource-record-data-soa}{number}
\defunx{resource-record-data-soa-retry}{resource-record-data-soa}{number}
\defunx{resource-record-data-soa-expire}{resource-record-data-soa}{number}
\defunx{resource-record-data-soa-minimum}{resource-record-data-soa}{number}
\begin{desc}
  A start of a zone of authority resource record. 
\end{desc}
The protocol specifies other possible values for the \texttt{resource-record-data}
field but we where no able to find test cases for them.


\defun{cache?}{thing}{\boolean}
\defunx{cache-answer}{cache}{dns-message}
\defunx{cache-ttl}{cache}{number}
\defunx{cache-time}{cache}{number}
\begin{desc}
  A cache data structure corresponds to a saved answer to a previous
  query. \ex{cache-answer} returns the saved message, \ex{cache-ttl}
  returns the time when the cache entry expires and \ex{cache-time}
  returns the time the entry was created.
\end{desc}

\section{Parsing \texttt{/etc/resolv.conf}}
\label{sec:dns-rc}

\defvar{resolv.conf-parse-error} {condition}
\defun{resolv.conf-parse-error?}{thing}{\boolean}
\begin{desc}
  The code signals the condition \var{resolv.conf-parse-error} if a
  parse error occurs while scanning \texttt{/etc/resolv.conf}. It is a
  subtype of the \var{dns-error} condition.
  \ex{resolv.conf-parse-error?} is the type predicate for this
  condition.
\end{desc}

\defun{resolv.conf}{}{{symbol$\rightarrow$string} alist}
\begin{desc}
  Returns the contents of \texttt{/etc/resolv.conv} as an alist with
  the possible keys \texttt{nameserver}, \texttt{domain},
  \texttt{search}, \texttt{sortlist} and \texttt{options}.
  
  Note that the library caches the contents of
  \texttt{/etc/resolv.conv} and \ex{resolv.conf} only really opens the
  file if its modification time is more recent than the modification
  time of the cache.
\end{desc}
\defun{parse-resolv.conf!}{}{\undefined}
\begin{desc}
  Parses the contents of \texttt{/etc/resolv.conv} and updates the
  internal cache of the library.
\end{desc}
\defun{dns-find-nameserver-list}{}{\fqdn list}
\begin{desc}
  Returns a list of name servers from \texttt{/etc/resolv.conf}
\end{desc}
\defun{dns-find-nameserver}{}{\fqdn}
\begin{desc}
  Returns the first name servers found in \texttt{/etc/resolv.conf}.
  \ex{dns-find-nameserver} raises \ex{no-nameservers} if
  \texttt{/etc/resolv.conf} does not contain a \texttt{nameserver}
  entry.
\end{desc}
\defun{domains-for-search}{}{\str{} list}
\begin{desc}
  Parses \texttt{/etc/resolv.conf} and extracts the domains specified
  by the \texttt{search} keyword.
\end{desc}


\section{IP Addresses as Dotted Strings}
%
\begin{description}
\item[Used files:] ip.scm
\item[Name of the package:] ips
\end{description}
%

The structure \ex{ips} provides a small set of procedures for turning
the human-readable form of IP addresses (``dotted strings'') into 32
bits numbers.

\defun{address32->ip-string}{\ipaddr}{ip-string}

\defun{ip-string->address32}{ip-string}{\ipaddr}

\defun{ip-string?}{string}{\boolean}
\begin{desc}
  Tests whether \var{string} is a valid dotted string for an IP
  address.
\end{desc}
%%% Local Variables: 
%%% mode: latex
%%% TeX-master: "man"
%%% End: 
