\chapter{POP3 Client}\label{cha:pop3}
%
The \ex{pop3} structure provides a client for the POP3 protocol that
allows access to email on a maildrop server.  It is often used in
configurations where users connect from a client machine which doesn't
have a permanent network connection or isn't always turned on,
situations which make local SMTP delivery impossible.  It is the most
common form of email access provided by ISPs.

Two types of authentication are commonly used.  The first, most basic
type involves sending a user's password in clear over the network, and
should be avoided.  (Unfortunately, many POP3 clients only implement this
basic authentication.)  The digest authentication system involves the
server sending the client a ``challenge'' token; the client encodes
this token with the pass phrase and sends the coded information to the
server.  This method avoids sending sensitive information over the
network.  Both methods are implemented by \ex{pop3}.

Once connected, a client may request information about the number and
size of the messages waiting on the server, download selected messages
(either their headers or the entire content), and delete selected
messages.

The procedures defined here raise an error detectable via
\ex{pop3-error?} upon protocol errors with the POP3 server.

\defun{pop3-connect}{[host-or-\sharpf] [login-or-\sharpf]
  [password-or-\sharpf] [log-port]}{connection}
\begin{desc}
  This procedure connects to the maildrop server named \var{host},
  and logs in using the provided login name and password.  Any of
  these can be omitted or \sharpf, in which case the procedure uses
  defaults: \ex{MAILHOST} for the host, and \ex{~/.netrc}-provided
  values for login and password.  If \var{log-port} is provided, the
  conversation to the server is logged to the specified output port.
  
  \ex{Pop3-connect} returns a value representing the connection to the
  POP3 server, to be used in the procedures below.
\end{desc}

\defun{pop3-stat}{connection}{number bytes}
\begin{desc}
  This returns the number of messages and the number of bytes waiting in the
  maildrop.
\end{desc}

Most of the following procedures accept a \var{msgid} argument which
specifies a message number, which ranges from 1 for the first message
to the number returned by \ex{pop3-stat}.

\defun{pop3-retrieve-message}{connection msgid}{headers message}
\begin{desc}
  This downloads message number \var{msgid} from the mailhost.
  It returns the headers as an alist of field names and bodies; the
  names are symbols, the bodies are strings.  (These are obtained
  using the \ex{rfc822} structure, see Section~\ref{cha:rfc822}.)
  The message is returned as a list of strings, each string
  representing a line of the message.
\end{desc}

\defun{pop3-retrieve-headers}{connection msgid}{headers}
\begin{desc}
  This downloads the headers of message number \var{msgid}.  It
  returns the headers in the same format as \ex{pop3-retrieve-message}.
\end{desc}

\defun{pop3-last}{connection}{msgid}
\begin{desc}
  This returns the highest accessed message-id number for the current
  session.  (This isn't in the RFC, but seems to be supported by several
  servers.)
\end{desc}

\defun{pop3-delete}{connection msgid}{undefined}
\begin{desc}
  This mark message number \var{msgid} for deletion.  The message will
  not be deleted until the client logs out.
\end{desc}

\defun{pop3-reset}{connection}{undefined}
\begin{desc}
  This marks any messages which have been marked for deletion.
\end{desc}

\defun{pop3-quit}{connection}{undefined}
\begin{desc}
  This closes the connection with the mailhost.
\end{desc}

\defun{pop3-error?}{thing}{boolean}
\begin{desc}
  This returns \sharpt{} if \var{thing} is a \ex{pop3-error} object,
  otherwise \sharpf.
\end{desc}

%%% Local Variables: 
%%% mode: latex
%%% TeX-master: "man"
%%% End: 
